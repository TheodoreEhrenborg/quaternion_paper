%%%%%%%%%%%%%%%%%%%%%%%%%%%%%%%%
%
%
%  Classifying Quaternion Identities
%  (journal article)
%  
%  by Theodore Ehrenborg
%
%
% 
% Last edited: May 7, 2019
%
%
%%%%%%%%%%%%%%%%%%%%%%%%%%%%%%%%%
%
%%%%%%%%%%%%%%%%%%%%%%%%%%%%%%%%%



%%%%%%%%%%%%%%%%%%%%%%%%%%%%%%%%%
%
% pdf settings
%
%%%%%%%%%%%%%%%%%%%%%%%%%%%%%%%%%
%
\pdfpagewidth=8.5truein
\pdfpageheight=11truein


%
%%%%%%%%%%%%%%%%%%%%%%%%%%%%%%%%%



\documentclass[12pt,table]{article}
%\usepackage{anyfontsize}
\usepackage{amssymb, amsmath, fullpage, amsthm}
\usepackage{mathrsfs}
\usepackage{tikz}
\usepackage[title]{appendix}
\usepackage{todonotes}
\usepackage{mathrsfs}
\usepackage{ gensymb }
\usepackage{enumerate}
\usepackage{caption}
\usepackage{xcolor}
\usepackage{diagbox}



\usetikzlibrary{math}




\parskip2mm

\newtheorem{theorem}{Theorem}[section]
\newtheorem{hypothesis}[theorem]{Hypothesis}
\newtheorem{lemma}[theorem]{Lemma}
\newtheorem{proposition}[theorem]{Proposition}
\newtheorem{corollary}[theorem]{Corollary}
\newtheorem{remarks}[theorem]{Remarks}
\newtheorem{conjecture}[theorem]{Conjecture}


\theoremstyle{definition}
\newtheorem{definition}[theorem]{Definition}
\newtheorem{example}[theorem]{Example}

\theoremstyle{remark}
\newtheorem{remark}[theorem]{Remark}

\hyphenation{Hurwitz}

\font\german = eufm10 scaled\magstep1
%\font\Cp = msbm10

\newcommand{\Ccc}{\mathbb C}
\newcommand{\Fff}{\mathbb F}
\newcommand{\Hhh}{\mathbb H}
\newcommand{\Nnn}{\mathbb N}
\newcommand{\Rrr}{\mathbb R}
\newcommand{\Sss}{\mathbb S}
\newcommand{\Zzz}{\mathbb Z}
\newcommand{\Lll}{\mathbb L}
\renewcommand{\Bbb}{\mathbb B}
\newcommand{\Ooo}{\mathbb O}
\newcommand{\Qqq}{\mathbb Q}


\newcommand{\vanish}[1]{}
\newcommand{\coveredby}{\prec}
\newcommand{\divides}{\mid}
\newcommand{\notdivides}{\nmid}
\newcommand{\timesdots}{\times \cdots \times}

\numberwithin{equation}{section}
%\usepackage[bindingoffset=0.2in,
%            left=1in,right=1in,top=1in,bottom=1in,
%            footskip=.25in]{geometry}%Sets margins
%\pagenumbering{gobble}%No page numbers




 


\DeclareMathOperator{\inv}{inv}
\DeclareMathOperator{\frst}{frst}
\DeclareMathOperator{\er}{er}
\DeclareMathOperator{\asc}{asc}
\DeclareMathOperator{\odd}{odd}
\DeclareMathOperator{\Imag}{Im}
\DeclareMathOperator{\Real}{Re}
\DeclareMathOperator{\N}{N}



\newcommand{\ascodd}{\asc_{\odd}}

\newcommand{\doubleprime}{\prime\prime}




\newcommand{\SSSS}{\mathfrak{S}}

\newcommand{\Gaussian}[2]{\genfrac{[}{]}{0pt}{}{#1}{#2}_q}

\begin{document}
%\begin{landscape}


\title{Classifying Quaternion Identities}


\author{\sc Theodore EHRENBORG%\thanks{Corresponding author:
%Department of Mathematics,
%University of Kentucky,
%Lexington, KY 40506-0027,
%USA,
%{\tt theodore.ehrenborg@gmail.com}.}
%\:\: 
%\:\:
}



\date{}
%%%\date{Last edited on \today}

\maketitle



\begin{abstract}
This number theory project investigates identities found by
multiplying together quaternions in $ \Lll[x,y,z,w] $, the
Lipschitz quaternions $ \Lll $ adjoined with the
indeterminates $x$, $y$, $z$, $w$.  Recall that quaternions
are $4$-dimensional complex numbers.  These identities provide
solutions to $ \sum_{j = 1}^{p} \tau_j ^ 2 = \left( \sum_{i = 1}^{m}
x_i ^ 2 \right) ^ n $. We present a rigorous definition that captures
the intuitive notion of when two such identities are equivalent. This
definition implies that the true structure of this problem involves the
group action of the direct product $ \mathfrak{S}_4^\pm \times \mathfrak{S}_4^\pm $.  Using
two complementary methods, we compute the number of equivalence
classes for $n = 1, 2, 3, 4,$ where $n$ is the number of
quaternion factors. We move to the case concerning products of complex
numbers, namely $ \Zzz[i][x,y] $. Using the fact that the
Gaussian integers are commutative under multiplication, we
characterize these equivalence classes, thus also providing an
enumeration.


\end{abstract}





\section{Introduction}






Recall that the \emph{complex numbers} $\Ccc$ are a two-dimensional
field extension of $\Rrr$.
Complex numbers take the form $x + iy$, where $i$ is defined to be the
square root of $-1$ and $x,y \in \Rrr$.
Addition is component-wise, and multiplication
follows from the distributive property and the fact that
$i^2 = -1$.
The \emph{conjugate} of a complex number $ z = x + iy$ is $ \overline{z} = x - iy$, and  
the \emph{norm} is $ \N(z) = z \overline{z} = x^2 + y^2$.  

The \emph{Gaussian integers} $ \Zzz[i] $ are a subring of $\Ccc$:
\[
\Zzz[i] = \{ x + iy: x,y \in \Zzz \}.
\]



Recall that
the \emph{quaternions} $\Qqq$ are a four-dimensional division ring extension of $\Rrr$.
Quaternions take the form $x + iy + jz + kw$, with
$x,y,z,w \in \Rrr$, and the elements $i$, $j$, $k$ satisfy the relations:
\begin{align*}
i^2 = j^2 = k^2 = -1
\\
ij = -ji = k
\\
jk = -kj = i
\\
ki = -ik = j.
\end{align*}
Addition is component-wise, and multiplication, which is
in general not commutative,
follows from the distributive property and the preceding relations.
The \emph{conjugate} of a quaternion $ \alpha = x + iy + jz + kw$
is $ \overline{\alpha} = x - iy - jz - kw$,  
and the \emph{norm} is 
$ \N( \alpha ) = \alpha \overline{\alpha} = x^2 + y^2 + z^2 + w^2$.  

The \emph{Lipschitz quaternions} $ \Lll $ are a subring of $\Qqq$:
\[
\Lll = \{ x + iy + jz + kw: x,y,z,w \in \Zzz \}.
\]



In the rest of this paper $x$, $y$, $z$, $w$ will
usually refer to indeterminates.


\todo[inline]{Talk about experimental data}

The following groups are an essential part of Definition~\ref{def:general}
and Definition~\ref{def:2D}.
\begin{definition}
The {\em symmetric group} $ \mathfrak{S}_n $ is the 
set of all permutations $ \pi = \pi_1 \cdots \pi_n $ 
of the $ n $ element set $ \{ 1, 2, \ldots, n \} $,
where $ \pi(i) = \pi_i $.
The {\em signed symmetric group} $ \mathfrak{S}_n^\pm $
is the set of all permutations $ \sigma = \sigma_1 \cdots \sigma_n$
of the set $ \{ \pm 1, \pm 2, \ldots, \pm n \} $ such that
$ | \sigma | = | \sigma_1 | \cdots |\sigma_n| \in \mathfrak{S}_n $.
\end{definition}

\todo[inline]{How to compose two elements in signed symmetric group?}

For a signed permutation $ \pi \in \mathfrak{S}_m^\pm $,
let $ \pi $ act on a polynomial in the 
$m$ variables $ x_1,x_2, \ldots, x_m $ by sending $ x_j $ to 
\[
\pi(x_j) =
\begin{cases}
x_{\pi_j} & \text{if } \pi_j > 0 \\
-x_{-\pi_j} & \text{if } \pi_j < 0
\end{cases}
\]

\begin{definition}
\label{def:general}
Fix $ p, m \in \Nnn $. 
Let $ \tau = ( \tau_1, \ldots, \tau_p) $
be a tuple of length $ p $ where 
$ \tau_i \in \Zzz[x_1,x_2, \ldots, x_m] $, where $ i = 1, \ldots, p $.
We define an equivalence relation, denoted by $ \simeq $, on $p$-tuples
$ \tau $ 
by taking the transitive closure of the following three relations:
\begin{itemize}
\item
$ ( \tau_1, \ldots, \tau_p) \simeq ( \tau'_1, \ldots, \tau'_p) $
if there exists a signed permutation $ \pi \in \mathfrak{S}_m^\pm $
acting on the $ m $ variables such that $ \pi( \tau_i ) = \tau'_i $,
 where $ i = 1, \ldots, p $.
\item
$ ( \tau_1, \ldots, \tau_p) \simeq ( \tau'_1, \ldots, \tau'_p) $
if there exists a permutation $ \sigma \in \mathfrak{S}_p $
such that $ \tau_{\sigma(i)} = \tau'_i $, where $ i = 1, \ldots, p $.
\item
$ ( \tau_1, \ldots, \tau_p) \simeq ( \pm \tau_1, \ldots, \pm \tau_p) $
\end{itemize}
\end{definition}


\begin{example}
Definition~\ref{def:general} implies that the following is true:
\begin{align*}
( xz,\: y^2,\: yz )  
&\simeq ( (-y)(-x),\: z^2,\: z(-x) ) \\
&\simeq ( (-y)(-x),\: z(-x),\: z^2 ) \\
&\simeq ( -(-y)(-x),\: z(-x),\: -z^2 ) 
\end{align*}
\end{example}



%Fix $ p \in \Zzz $. 
We are interested in counting the number of
equivalence classes of the set of all tuples
 $ \tau = ( \tau_1, \ldots, \tau_p) $
where
%the sum of the squares of the elements of $ \tau $ is
\[
\sum_{j = 1}^{p}  \tau_j ^ 2  
= 
\left( \sum_{i = 1}^{m}  x_i ^ 2  \right) ^ n 
\] 
This problem is most easily attacked when we
view $ \left( \sum_{i = 1}^{m}  x_i ^ 2  \right) ^ n $
as the norm of a product of complex numbers or quaternions.
Thus we will focus on the cases where $ p = m = 2 $ and where $ p = m = 4 $.
The general problem can also be viewed as finding the disjoint orbits of 
various tuples, where the group action is $ \mathfrak{S}_p^\pm \times \mathfrak{S}_m^\pm $.



\begin{example}

Consider the case where $ p = m = 2 $ and $ n = 2$.
The following two identities are representatives from 
the two different equivalence classes in this case. These identities
were generated by a product of complex numbers. 

\noindent
Identity i:
\begin{equation*}
(x + iy)(x + iy) = (x^2 - y^2 ) + i(2xy) 
\end{equation*}
Taking norms gives the identity
\begin{equation}
    (x^2 - y^2 )^2 + (2xy)^2 
    = (x^2 + y^2)^2.
\end{equation}
Identity ii:
\begin{equation*}
    (x + iy )(x - iy )
    = (x^2 + y^2 ) + i(0)  
\end{equation*}
Taking norms gives the identity
\begin{equation}
    (x^2 + y^2 )^2 + (0)^2
    = (x^2 + y^2 )^2.
\end{equation}
\end{example}



\begin{example}
The following two identities are representatives from 
the two different equivalence classes in the
case where $ p = m = 2 $ and $ n = 3$.
These identities
were generated by a product of complex numbers. 

\noindent
Identity iii:
\begin{align*}
    (x + iy)(x + iy)(x + iy) 
    = x(x^2 - 3y^2) + i(  y(3x^2 - y^2) )  
    \end{align*}
Taking norms gives the identity
    \begin{align}
    (x(x^2 - 3y^2))^2 + (  y(3x^2 - y^2) )^2  
    = (x^2 + y^2)^3.
    \end{align}
Identity iv:
    \begin{align*}
    (x + iy )(x + iy)(x - iy ) 
    = x(x^2 + y^2 ) + i(y(x^2 + y^2))  
    \end{align*}
Taking norms gives the identity
    \begin{align}
    ( x(x^2 + y^2) )^2 + ( y(x^2 + y^2) )^2 
    = (x^2 + y^2 )^3.
    \end{align}
\end{example}







\section{The case where $ p = m = 2$}


In the case where $ p = m = 2$, Definition~\ref{def:general} has an alternate form.



\begin{definition}
\label{def:2D}

Let $ h(z), h'(z) \in \Zzz [i][x,y] $,
where $ z = x + iy $.
Let $ M $ be the following set of mappings: 
\[
M = \{ z \mapsto uz \text{ : } u \in \{ \pm 1, \pm i \} \}  
\cup \{ z\mapsto u \bar{z} \text{ : } u \in \{ \pm 1, \pm i \} \}.  
\]
%Write $ f(x,y) + i \cdot g(x,y) $ and $ f'(x,y) + i \cdot g'(x,y) $ as  $ h(z) $ and  $ h'(z) $, respectively.
We say $ h(z) \simeq h'(z) $ when there exist mappings $ \varphi, \varphi' \in M $
such that $ \varphi( h( \varphi'( z ) ) )  = h'(z) $.



\end{definition}

\begin{lemma}
The relation $ \simeq $ is an equivalence relation.
\end{lemma}

\begin{proof}
The relation $ \simeq $ satisfies the three conditions of an equivalence relation.
\begin{enumerate}
\item Reflexive Property: If we choose $ \mu $ and $ \mu'$ to be the identity map $ z \mapsto z $, 
then $ h(z) \simeq h(z) $.

\item Symmetric Property: Let  $ h(z) \simeq h'(z) $, that is, there exist mappings 
$ \mu, \mu' \in M $
such that $ \mu( h( \mu'( z ) ) )  = h'(z) $.
$M$ is isomorphic to the signed symmetric group $ \mathfrak{S}_2^\pm $, so every mapping in $M$
has an inverse in $M$. As  $ \mu^{-1}( h'( (\mu')^{-1}( z ) ) )  = h(z) $,
we have $ h'(z) \simeq h(z) $.

\item Transitive Property: Let $ a(z) \simeq b(z) $ and $ b(z) \simeq c(z) $. By the 
Symmetric Property, we have $ c(z) \simeq b(z) $. Thus there exist mappings
$ \mu, \mu', \nu, \nu' \in M $ such that $ \mu( a( \mu'( z ) ) )  = b(z) $ and 
$ \nu( c( \nu'( z ) ) )  = b(z) $. As a result:
\[
\mu( a( \mu'( z ) ) ) = \nu( c( \nu'( z ) ) )  
\]
Thus we have: 
\[
a(  z  ) = \mu^{-1}( \nu( c( (\mu')^{-1}( \nu'( z ) ) ) ) )
\]
This means that $ c(z) \simeq a(z) $ or $ a(z) \simeq c(z) $.
\end{enumerate}
\end{proof}

Recall that given $ f(x,y), g(x,y) \in \Zzz[x,y] $,
we can find $ h(z) \in \Zzz [i][x,y] $
such that
$ z = x + iy $
and
$ h(z) = f(x,y) + i \cdot g(x,y) $. 
The converse is also true.

\todo[inline]{Does $\simeq$ apply to complex numbers/quaternions or just tuples? }

\begin{lemma}
Let $ h(z) \simeq h'(z) $, where $ h(z), h'(z) \in \Zzz[i][x,y] $
and $ z = x+ iy $.
Suppose $ ( x^2 + y^2 ) ^ u \divides h(z) $,
where $ u \in \Nnn $. Then $ ( x^2 + y^2 ) ^ u \divides h'(z) $.
\end{lemma}
   
\begin{proof}
Let $ h(z) = ( x^2 + y^2 ) ^ u  p(z) $, where $ p(z) \in \Zzz [i][x,y] $. 
Whatever mappings we apply to $h$ and $z$
to obtain $ h'(z) $, we also apply to the factors of $ h(z) $. 
No mapping will remove the factor of $ ( x^2 + y^2 ) ^ u $.

\end{proof}

\vanish{
\begin{corollary}
Let $ h(z), h'(z) \in \Zzz[i][x,y] $ with $ z = x+ iy $.
\[
h(z) \simeq h'(z) 
\implies 
( \forall u \in \Nnn \cup \{ 0 \},  ( x^2 + y^2 ) ^ u \divides h(z)  \iff ( x^2 + y^2 ) ^ u \divides h'(z)  )
\]
\end{corollary}
}

\begin{corollary}
\label{corollary_equivalence}
Let $ h(z), h'(z) \in \Zzz[i][x,y] $ with $ z = x+ iy $.
Suppose $ h(z) \simeq h'(z) $. Then for all  $ u \in \Nnn \cup \{ 0 \} $,
the following two conditions are
equivalent:
\begin{enumerate}[i.]
\item $ ( x^2 + y^2 ) ^ u \divides h(z) $
\item $ ( x^2 + y^2 ) ^ u \divides h'(z)  ) $
\end{enumerate}
\end{corollary}




\begin{theorem}
Consider the set of all $2$-tuples $ \tau = ( \tau_1, \tau_2 )$ where 
$
  \tau_1 ^ 2   +   \tau_2 ^ 2   
= 
\left(  x_1 ^ 2 + x_2 ^ 2  \right) ^ n 
$.
The number of equivalence classes within this set 
is exactly  $ \left\lfloor \frac{n}{2} \right\rfloor + 1 $. 
Moreover, each equivalence class contains a tuple
of the form $ ( \Real( \beta ) , \Imag( \beta ) ) $,
where $ \beta = (x + iy)^j (x -  iy)^{n-j} $,
with $ j $
being one of $ 0, 1, 2, \ldots, \left\lfloor \frac{n}{2} \right\rfloor $.
\end{theorem}



\begin{proof}
Suppose 
\[ (x^2 + y^2)^n = f(x,y)^2 + g(x,y)^2 \]
where $ f(x,y), g(x,y) \in \Zzz[x,y] $.
Then  
\[ (x + iy)^n (x -  iy)^n = ( f(x,y) + i \cdot g(x,y) ) ( f(x,y) - i \cdot g(x,y) ) .\]
Since $ \Zzz [i][x,y] $ is a unique factorization domain,
we can factor this as
\[ f(x,y) + i \cdot g(x,y) = c \cdot (x + iy)^j (x -  iy)^k \]
and
\[ f(x,y) - i \cdot g(x,y) = d \cdot (x + iy)^r (x -  iy)^s,\]
where $ j, k, r, s \in \Nnn \cup \{0\} $ and
$c \cdot d = 1 $ with $ c,d \in \{ \pm 1 , \pm i \} $. 
We know $ j + r = n $ and $ k + s = n $, as well as
(by taking norms) $ j + k = n $ and $ r + s = n $.
Thus $k = r$ and $j = s$.

Clearly $ f(x,y) + i \cdot g(x,y) \simeq  f(x,y) - i \cdot g(x,y) $.
Since 
\[ 
f(x,y) + i \cdot g(x,y) = c \cdot (x + iy)^j (x -  iy)^{n-j} 
\]
and 
\[ 
f(x,y) - i \cdot g(x,y) = d \cdot (x + iy)^{n-j} (x -  iy)^j,
\]
each equivalence class contains a representative with $ j \leq n - j
$.  As $ 2j \leq n $, we have $ j \leq \frac{n}{2} $, so $ j $
is one of $ 0, 1, 2, \ldots, \left\lfloor \frac{n}{2} \right\rfloor $. This shows
that there are at most $ \left\lfloor \frac{n}{2} \right\rfloor + 1 $
equivalence classes. Now we show that there are at least that many.



Consider $ (x + iy)^u (x -  iy)^{n-u} $  and $ (x + iy)^v (x -  iy)^{n-v} $,
where $ u \neq v $ and $u,v \leq [ \frac{n}{2} ] $. We have
\begin{align*}
(x + iy)^u (x -  iy)^{n-u} &= (x^2 + y^2)^u (x -  iy)^{n-2u}
\\
(x + iy)^v (x -  iy)^{n-v} &= (x^2 + y^2)^v (x -  iy)^{n-2v}
\end{align*}
Without loss of generality, we may assume $ u > v $. 
Since $ \Zzz [i][x,y] $ is a unique factorization domain,
$ (x^2 + y^2) \notdivides (x -  iy)^e $ for $ e \in \Nnn \cup \{ 0 \} $.
Thus $ (x^2 + y^2)^u \divides (x + iy)^u (x -  iy)^{n-u} $
but $ (x^2 + y^2)^u \notdivides (x + iy)^v (x -  iy)^{n-v} $.

By Corollary~\ref{corollary_equivalence},
$ (x + iy)^u (x -  iy)^{n-u} \not\simeq (x + iy)^v (x -  iy)^{n-v} $, 
so the $ \left\lfloor \frac{n}{2} \right\rfloor + 1 $ representatives arise from
distinct equivalence classes.
Thus there are $ \left\lfloor \frac{n}{2} \right\rfloor + 1 $ equivalence classes, 
one each for $ u = 0, 1, 2, \ldots,  \left\lfloor \frac{n}{2} \right\rfloor $.
\end{proof}







\begin{figure}[h]
\label{fig:2D}

\begin{center}
\begin{tikzpicture}

\tikzmath{
\hash = 0.1;
\height = 6;
\length = 10;
\scale = 1;
}


\draw[thick] (0,0) -- (\scale * \length,0);
\fill (\scale * \length/2, -1) circle (0cm) node[anchor=north] {\Large $ n $ };

\draw[thick] (0,0) -- (0,\scale * \height);
%\fill (-2, \scale * \height/2 ) circle (0cm) node[rotate=90,anchor=south] {\Large  Number of };
%\fill (-1, \scale * \height/2 ) circle (0cm) node[rotate=90,anchor=south] {\Large  equivalence classes };
\fill (-1, \scale * \height/2 ) circle (0cm) node[anchor=east] {\Large  $ \kappa $ };


\foreach \y in {1, ..., \height}
    \draw[thick] ( -\hash, \scale * \y ) node[anchor=east] {\Large \y} -- ( \hash, \scale * \y );

\foreach \x in {1, ..., \length}
{
   \draw[thick] ( \scale * \x, -\hash ) node[anchor=north] {\Large \x} -- ( \scale * \x, \hash ) ;
   \fill (\scale * \x, { \scale *  floor( \x / 2 ) + \scale } ) circle (.1cm);
}
  

\end{tikzpicture}

\end{center}

\caption{
The number of equivalence classes $ \kappa $
when $p = m = 2$ is $ \left\lfloor \frac{n}{2} \right\rfloor + 1 $ .
 }
\end{figure}



\begin{corollary}

The number of solutions $ (f, g) $ to
\[
f^2 + g^2 = (x^2 + y^2)^n,
\]
where $ f, g \in \Zzz[x,y] $, is $4n + 4$.

\end{corollary}
\begin{proof}
It is sufficient to count the orbits of a representative from each
equivalence class, where the group action is
$ \mathfrak{S}_2^\pm \times \mathfrak{S}_2^\pm $.

\noindent
Case I: $ n = 2z $ for some $ z \in \Zzz $.

There are $ z + 1 $ equivalence classes. We choose $ z + 1 $
representatives of the form $ (x - iy)^j (x + iy)^k $, with
$ 0 \leq j \leq \frac{n}{2}$ and $ j + k = n $.
When $ j = k $, the representative
is $ (x - iy)^z (x + iy)^z = ( x^2 + y^2 ) ^ z $, which
gives the tuple $ ( x^2 + y^2, 0 ) $, which has an orbit of order 4.
Otherwise, we write the representative as $ (x + iy)^{k-j} (x^2 + y^2)^j $.
As $ k - j $ is even, let $ k - j = 2w $. We have
\begin{align}
\label{equation_first_binomial}
(x + iy) ^ {2w} &= x ^ {2w} - \binom{2w}{2}  x^{2w - 2 } y^2
+ \cdots + \binom{2w}{2w-2} x^2 y^{2w - 2} (-1) ^ {w - 1}
+  y^{2w} (-1) ^ {w} \nonumber
\\
&+ i \left( \binom{2w}{1} x ^ {2w - 1} y + \cdots +
\binom{2w}{2w-1}  x y ^ {2w -1} (-1)^{w-1} \right).
\end{align}
As a tuple, this has two entries, where the first is the real part
of equation~\ref{equation_first_binomial} and the second entry
is the imaginary part.
The order of the stabilizer is 8 --- the two variables can be rearranged and their
signs switched. This may require the sign of an entire expression to be switched,
depending on the parity of $ w $.
As $ | \mathfrak{S}_2^\pm \times \mathfrak{S}_2^\pm | = 64$, the order of each orbit
is also 8. 

Thus in Case I there are $ 8z + 4 = 4 (n + 1 ) $ solutions
to $ f^2 + g^2 = (x^2 + y^2)^n $.



\noindent
Case II: $ n = 2z + 1$ for some $ z \in \Zzz $.

There are $ z + 1 $ equivalence classes. We choose $ z + 1 $
representatives of the form $ (x - iy)^j (x + iy)^k $, with
$ 0 \leq j \leq \frac{n}{2}$ and $ j + k = n $.
It is impossible to
choose $ j $ such that $ \Imag\left( (x - iy)^j (x + iy)^k \right) = 0$.
We write the representative as $ (x + iy)^{k-j} (x^2 + y^2)^j $.
As $ k - j $ is odd, let $ k - j = 2w + 1 $. We have
\begin{align}
\label{equation_second_binomial}
(x + iy) ^ {2w + 1} &= x ^ {2w + 1} - \binom{2w+1}{2}  x^{2w - 1 } y^2
+ \cdots + \binom{2w+1}{2w} x y^{2w} (-1) ^ {w} \nonumber
\\
&+ i \left( \binom{2w+1}{1} x ^ {2w} y + \cdots +
\binom{2w+1}{2w-1}  x^2 y ^ {2w -1} (-1)^{w-1}
+  y^{2w+1} (-1) ^ {w}
\right).
\end{align}
As a tuple, this has two entries, where the first is the real part
of equation~\ref{equation_second_binomial} and the second entry
is the imaginary part.
The order of the orbit of the tuple under the group action is 8, which can be
verified in a similiar manner to Case I.

Thus the number of solutions in Case II is $ 8(z + 1) = 4 ( 2w + 1) + 4 = 4n + 4 $.
\end{proof}




\section{The case where $p = m = 4$ and $n = 2$}





\begin{table}[h]

\label{table:4D}


\begin{center}


\begin{tabular}{ c | c }
 $ n $ & $ \kappa $ \\
\hline\hline
 1 & 1 \\
\hline
 2 & 8 \\
\hline
 3 & 48 \\
\hline
 4 & 965 
\end{tabular}





\end{center}
\caption{
The conjectured number of equivalence
classes $ \kappa $  
in the case where $p = m = 4$.
 }

\end{table}






\begin{table}[h]

\label{table_barcode}


\begin{center}


\begin{tabular}{ r || c | c | c | c | c | c | c | c | }
\diagbox{Value}{Identity} & \ref{Identity_1} & \ref{Identity_2}
& \ref{Identity_3} & \ref{Identity_4}
& \ref{Identity_5} & \ref{Identity_6}
& \ref{Identity_7} & \ref{Identity_8}
\\
\hline\hline
$0^2 + 0^2 + 0^2 + 0^2 = 0^2$ & \cellcolor{gray!50} & \cellcolor{gray!50}
& \cellcolor{gray!50} & \cellcolor{gray!50}
& \cellcolor{gray!50} & \cellcolor{gray!50}
& \cellcolor{gray!50} & \cellcolor{gray!50}
\\
\hline
$0^2 + 0^2 + 0^2 + 1^2 = 1^2$ & \cellcolor{gray!50} & \cellcolor{gray!50}
& \cellcolor{gray!50} & \cellcolor{gray!50}
& \cellcolor{gray!50} & \cellcolor{gray!50}
& \cellcolor{gray!50} & \cellcolor{gray!50}
\\
\hline
$0^2 + 0^2 + 0^2 + 2^2 = 2^2$ & \cellcolor{gray!50} & \cellcolor{gray!50}
& \cellcolor{gray!50} & \cellcolor{gray!50}
& \cellcolor{gray!50} & \cellcolor{gray!50}
& \cellcolor{gray!0} & \cellcolor{gray!0}
\\
\hline
$0^2 + 0^2 + 0^2 + 3^2 = 3^2$ & \cellcolor{gray!50} & \cellcolor{gray!0}
& \cellcolor{gray!50} & \cellcolor{gray!50}
& \cellcolor{gray!50} & \cellcolor{gray!0}
& \cellcolor{gray!50} & \cellcolor{gray!50}
\\
\hline
$0^2 + 0^2 + 0^2 + 4^2 = 4^2$ & \cellcolor{gray!0} & \cellcolor{gray!50}
& \cellcolor{gray!50} & \cellcolor{gray!0}
& \cellcolor{gray!50} & \cellcolor{gray!0}
& \cellcolor{gray!0} & \cellcolor{gray!50}
\\
\hline
$0^2 + 1^2 + 2^2 + 2^2 = 3^2$ & \cellcolor{gray!50} & \cellcolor{gray!50}
& \cellcolor{gray!0} & \cellcolor{gray!50}
& \cellcolor{gray!50} & \cellcolor{gray!50}
& \cellcolor{gray!50} & \cellcolor{gray!50}
\\
\hline
$1^2 + 1^2 + 1^2 + 1^2 = 2^2$ & \cellcolor{gray!0} & \cellcolor{gray!0}
& \cellcolor{gray!0} & \cellcolor{gray!50}
& \cellcolor{gray!50} & \cellcolor{gray!50}
& \cellcolor{gray!50} & \cellcolor{gray!50}
\\
\hline
$2^2 + 2^2 + 2^2 + 2^2 = 4^2$ & \cellcolor{gray!50} & \cellcolor{gray!0}
& \cellcolor{gray!0} & \cellcolor{gray!50}
& \cellcolor{gray!0} & \cellcolor{gray!50}
& \cellcolor{gray!50} & \cellcolor{gray!0}
\\
\hline
\end{tabular}





\end{center}
\caption{
Caption goes here
 }

\end{table}





\begin{theorem}
\label{theorem_8_classes}
In the case where  $p = m = 4$ and $n = 2$,
the following identities 
come from 8 different
equivalence classes.

    \begin{align}
    \label{Identity_1}
    (x^2 - y^2 - z^2 - w^2 )^2 + (2xy)^2 + (2xz)^2 + (2xw)^2 
    &= (x^2 + y^2 + z^2 + w^2)^2 
\end{align}
\begin{align}
\label{Identity_2}
    (x^2 - y^2 - z^2 + w^2 )^2 + (2xy - 2zw)^2 + (2xz + 2yw)^2 + (0)^2 
    &= (x^2 + y^2 + z^2 + w^2)^2
\end{align}
\begin{align}
\label{Identity_3}
    (x^2 + y^2 + z^2 + w^2 )^2 + (0)^2 + (0)^2 + (0)^2 
    &= (x^2 + y^2 + z^2 + w^2)^2
\end{align}
\begin{align}
\label{Identity_4}
    (x^2 - y^2 - 2zw )^2 + (2xy + z^2 - w^2)^2 
    + (xz - yz + xw + yw)^2\nonumber
    \\
    + (xz - yz + xw + yw)^2 
        &= (x^2 + y^2 + z^2 + w^2)^2
\end{align}
\begin{align}
\label{Identity_5}
    (x^2 - y^2 )^2 + (2xy - z^2 - w^2)^2 
        + (xz + yz + xw + yw)^2 \nonumber
        \\
        + (-xz - yz + xw + yw)^2 
    &= (x^2 + y^2 + z^2 + w^2)^2
\end{align}
\begin{align}
\label{Identity_6}
    (x^2 + y^2 )^2 + (- z^2 - w^2)^2 
        + (xz + yz + xw - yw)^2  \nonumber
        \\
      + (-xz + yz + xw + yw)^2
    &= (x^2 + y^2 + z^2 + w^2)^2
\end{align}
\begin{align}
\label{Identity_7}
    (x^2 - yz - yw - zw )^2 + (xy + xz + yz - w^2)^2 
        + (-y^2 + xz + xw + zw)^2 \nonumber
        \\
        + (xy - z^2 + xw + yw)^2 
    = (x^2 + y^2 + z^2 + w^2)^2
\end{align}
\begin{align}
\label{Identity_8}
    (x^2 - yz + yw - zw )^2 + (xy + xz - yz - w^2)^2 
        + (y^2 + xz + xw + zw)^2 \nonumber
        \\
        + (xw - xy - z^2  + yw)^2 
            = (x^2 + y^2 + z^2 + w^2)^2
    \end{align}



Furthermore,
each identity
is generated by a product of quaternions. 
\end{theorem}
\begin{proof}
Identity~\ref{Identity_1} follows from taking the norm of the quaternion product
    \begin{align*}
    (x + iy + jz + kw)(x + iy + jz + kw) 
    = (x^2 - y^2 - z^2 - w^2 ) + i(2xy) + j(2xz) + k(2xw).
    \end{align*}
Identity~\ref{Identity_2} follows from taking the norm of the quaternion product    
    \begin{align*}
    &(x + iy + jz + kw)(x + iy + jz - kw) \\
    &= (x^2 - y^2 - z^2 + w^2 ) + i(2xy - 2zw) + j(2xz + 2yw) + k(0). 
    \end{align*}
Identity~\ref{Identity_3} follows from taking the norm of the quaternion product        
    \begin{align*}
    &(x + iy + jz + kw)(x - iy - jz - kw) \\
    &= (x^2 + y^2 + z^2 + w^2 ) + i(0) + j(0) + k(0).
    \end{align*}
Identity~\ref{Identity_4} follows from taking the norm of the quaternion product        
    \begin{align*}
    &(x + iy + jz + kw)(x + iy + jw + kz) \\
    &= (x^2 - y^2 - 2zw ) + i(2xy + z^2 -w^2) \\
        &+ j(xz - yz + xw + yw) + k(xz - yz + xw + yw).
    \end{align*}
Identity~\ref{Identity_5} follows from taking the norm of the quaternion product            
    \begin{align*}
    &(x + iy + jz + kw)(x + iy + jw - kz) \\
    &= (x^2 - y^2 ) + i(2xy - z^2 - w^2) \\
        &+ j(xz + yz + xw + yw) + k(-xz - yz + xw + yw).
    \end{align*}
Identity~\ref{Identity_6} follows from taking the norm of the quaternion product
    \begin{align*}
    &(x + iy + jz + kw)(x - iy + jw - kz) \\
    &= (x^2 + y^2 ) + i(- z^2 - w^2) \\
        &+ j(xz + yz + xw - yw) + k(-xz + yz + xw + yw).
    \end{align*}
Identity~\ref{Identity_7} follows from taking the norm of the quaternion product
    \begin{align*}
    &(x + iy + jz + kw)(x + iz + jw + ky) \\
    &= (x^2 - yz - yw - zw ) + i(xy + xz + yz - w^2) \\
        &+ j(-y^2 + xz + xw + zw) + k(xy - z^2 + xw + yw).
    \end{align*}
Finally, Identity~\ref{Identity_8} follows from taking the norm of the quaternion product    
    \begin{align*}
    &(x + iy + jz + kw)(x + iz + jw - ky) \\
    &= (x^2 - yz + yw - zw ) + i(xy + xz - yz - w^2) \\
        &+ j(y^2 + xz + xw + zw) + k(xw - xy - z^2  + yw). \tag*{\qedhere}
    \end{align*}
\end{proof}


\begin{conjecture}
In the case where  $p = m = 4$ and $n = 2$,
there are exactly 8 equivalence classes, which have representatives
given in Theorem~\ref{theorem_8_classes}.
\end{conjecture}


\section{Conjecture for $p = m = 4$ and $ n \in \Nnn $}
This section represents joint work with Professor David Leep.

We conjecture that when $p = m = 4$ and $ n \in \Nnn $,
all identities of the form
$
\tau_1 ^ 2  + \tau_2 ^ 2  + \tau_3 ^ 2  + \tau_4 ^ 2  
= 
\left(  x_1 ^ 2 + x_2 ^ 2 + x_3 ^ 2 + x_4 ^ 2
\right) ^ n 
$
can be written as
a product of Lipschitz quaternions.
This is made more precise in
Conjecture~\ref{conjecture_Ehrenborg-Leep}.

\begin{conjecture}
Let $ \alpha \in \Lll[x,y,z,w] $.
If the norm $ \N(\alpha) $ has a nontrivial
factorization over $ \Zzz[x,y,z,w] $,
then $ \alpha $ has a nontrivial
factorization over $ \Lll[x,y,z,w] $.
\end{conjecture}
The preceding conjecture implies the following conjecture.
\begin{conjecture}
\label{conjecture_Ehrenborg-Leep}
Let $ (a, b, c, d) $ be a tuple such that
\[
a^2 + b^2 + c^2 + d^2 = (x^2 + y^2 + z^2 + w^2)^n,
\] 
where $ a, b, c, d \in \Zzz[x,y,z,w]$.
Let $\alpha = a + bi + cj + dk $.
Then $ \alpha = \beta_1 \beta_2 \cdots \beta_n $,
where $ \beta_u \in \Lll[x,y,z,w] $ for $u$ in $1, \ldots, n$.
Moreover, let $ \beta_u = a' + b'i + c'j + d'k $. Then
the tuple $ (a', b', c', d') \simeq (x, y, z, w) $.
\end{conjecture}



\section{Proof of Conjecture~\ref{conjecture_Ehrenborg-Leep} when $n=1$}
I have proved Conjecture~\ref{conjecture_Ehrenborg-Leep}
in the case where $n = 1$.


Recall that $ \theta \in \Zzz[x,y,z,w] $ is a \emph{monomial} if $ \theta $
is of the form $Cx^{e_1}y^{e_2}z^{e_3}w^{e_4}$, where $ C \in \Zzz $
and $ e_1,e_2,e_3,e_4 \in \Nnn \cup \{0\} $.
Two monomials $ u$ and $ v $ are \emph{similar}, denoted  $ u \sim v $,
if they are equal or only differ in their
coefficient.
We denote the degree of  $ \theta $ with repect to $x$ 
by  $\deg_x \theta = e_1 $. We use analogous notation for the
other three variables.

\begin{remark}
Any element of $ \Zzz[x,y,z,w] $ is a sum of monomials.
\end{remark}

\begin{lemma}
If $a, b, c, d \in \Zzz[x,y,z,w]$, and
\[
a^2 + b^2 + c^2 + d^2 = x^2 + y^2 + z^2 + w^2,
\]
then $ (a, b, c, d ) \simeq (x, y, z, w )$.
\end{lemma}
\begin{proof}
Write each of $ a,b,c,d $ as sums of monomials, that is, 
\begin{align*}
&a = \theta_{11} + \theta_{12} + \cdots + \theta_{1\alpha}
\\
&b = \theta_{21} + \theta_{22} + \cdots + \theta_{2\beta}
\\
&c = \theta_{31} + \theta_{32} + \cdots + \theta_{3\gamma}
\\
&d = \theta_{41} + \theta_{42} + \cdots + \theta_{4\delta}
\end{align*}
where the monomial $ \theta_{ij}$ is not similar
to the monomial $\theta_{ik}$ for $ j \neq k$.

Let $S$ be the \emph{multiset} of all the preceding $ \theta $'s.
We select a particular element~$ \omega $
from $S$ in the following manner.
Let $ S_1 $ be the set of all $ \theta \in S  $ such that
$ \deg_x \theta $ is the maximum possible for all $ \theta \in S $.
Let $ S_2 $ be the set of all $ \theta \in S_1  $ such that
$ \deg_y \theta $ is the maximum possible for all $ \theta \in S_1 $.
Let $ S_3 $ be the set of all $ \theta \in S_2  $ such that
$ \deg_z \theta $ is the maximum possible for all $ \theta \in S_2 $.
Finally, let $ S_4 $ be the set of all $ \theta \in S_3  $ such that
$ \deg_w \theta $ is the maximum possible for all $ \theta \in S_3 $.
Let $ \omega $ be an element of $ S_4 $. 
We can write $ \omega = Cx^{e_1}y^{e_2}z^{e_3}w^{e_4}$, where $ C \in \Zzz $
and $ e_1,e_2,e_3,e_4 \in \Nnn \cup \{0\} $.

We first assume that $\deg_x \omega > 1 $.
In the expression  
$a^2 + b^2 + c^2 + d^2$, there must be at least one monomial similar to $ \omega^2 $
that cancels the term $ \omega^2 = C^2x^{2e_1}y^{2e_2}z^{2e_3}w^{2e_4}$.
Let one of these monomials be $ \psi $. Assume for the moment that $ \psi $
is not formed from squaring a monomial in $S$, but rather
from a product of two nonsimilar monomials.
In other words, $ \psi = \psi' \cdot \psi''$, where $ \psi' \nsim \psi'' $
and $ \psi' , \psi'' \in S $. Since $\omega$ was chosen to have maximal $x$-degree,
we have $\deg_x \psi' \leq \deg_x \omega$ and
$\deg_x \psi'' \leq \deg_x \omega$.
As $\omega^2 \sim \psi = \psi' \cdot \psi''$,
we have $ \deg_x \omega^2 = \deg_x \psi' + \deg_x \psi''$
and immediately have $ \deg_x \omega = \deg_x \psi' = \deg_x \psi'' $.
Thus $ \psi' , \psi'' \in S_1 $. Continuing in this manner
with the other variables, we
conclude that $ \psi' $ and $ \psi'' $ are similar monomials.
This contradicts the fact that $ \psi' \nsim \psi'' $.
Hence $ \psi $ was
formed by the square of a monomial in $ S $. However, this implies the coefficient
of $ \psi $ is positive. Furthermore, the coefficient of
any monomial similar to $ \omega^2 $ must be positive,
so $ \omega^2 $ cannot be cancelled. This is
a contradiction, so $ \deg_x \omega \leq 1 $. Thus no monomial in $S$ has $x$-degree more
than $1$. In an analogous way, we can show that no monomial in $S$ has $y$-degree, $z$-degree,
or $w$-degree more than $1$.

By setting the three variables $y$, $z$, $w$ equal to zero, we see that $ S $ contains
exactly one monomial similar to $ x $, namely $ x$ or $ -x $. The same is true for
the other three variables. Moreover, there are no constants in $ S $.

Assume that $\theta_{11} \sim x$ and $\theta_{12} \sim y$.
Then $a^2$ contains
a monomial similar to $ xy $, which can only be cancelled by another product of monomials similar to
$ x $ and $y$. However, we showed that $ S $ has no such other monomials.
Thus each coordinate $ (a, b, c, d) $ contains exactly one of $ \pm x$, $\pm y$, $\pm z$, $\pm w $.


Assume that the coordinate that contains $ \pm x $ also contains
another monomial $ A = Dx^{f_1}y^{f_2}z^{f_3}w^{f_4}$, where $ D \in \Zzz $
and $ f_1,f_2,f_3,f_4 \in \{0,1\} $. Then $ A^2 $ has degrees (with respect to
each variable) of either 0 or 2. To cancel $ A^2 $, we must find a product of two
monomials in $ S $ that is similar to $ A^2 $. As no monomial in $ S $ has a degree
(with respect to any variable) of two, any monomial similar to $ A^2 $ must itself be
a square of a monomial similar to $ A $. In this case, the coefficients are
positive and will not cancel. Thus there can be no such monomial $ A $, so the only monomials
in $ S $ are the four desired ones.
\end{proof}

\section{Enumerative Consequences of Conjecture~\ref{conjecture_Ehrenborg-Leep}} 

\begin{theorem}
\label{theorem_counting_4D}
Assuming Conjecture~\ref{conjecture_Ehrenborg-Leep} holds,
then the number of solutions $ (a, b, c, d) $ to
\begin{equation}
\label{equation_4D}
a^2 + b^2 + c^2 + d^2 = (x^2 + y^2 + z^2 + w^2)^n,
\end{equation}
where $ a, b, c, d \in \Zzz[x,y,z,w]$,
is at most
\[
8 \cdot \frac{47^{n+1} - 1}{46}
=  8 \cdot \left\lfloor \frac{ 47^{n+1} }{46} \right\rfloor
=  8 \sum_{i = 0}^n{47^n}.
\]
\end{theorem}
\begin{proof}
If Conjecture~\ref{conjecture_Ehrenborg-Leep} holds, any
solution
$ (a, b, c, d) $ can be viewed as a quaternion
$ a + bi + cj + dk = \beta_1 \beta_2 \cdots \beta_n $
where the $\beta_i$ are quaternions from the following set:
\[
T_1 = \{ \pm a' \pm b'i \pm c'j \pm d'k
\text{ : } \{ a', b', c', d' \} = \{ x, y, z, w \} \}. 
\]
Note that $ | T_1 | = 2^4 4! = 384 $.
Thus there are at most $ 384^n $ unique solutions to equation~\ref{equation_4D}.
Using the fact that
\[
\beta_1 \beta_2 \cdots \beta_i \beta_{i+1} \cdots \beta_n
=
\beta_1 \beta_2 \cdots \beta_i u u^{-1} \beta_{i+1} \cdots \beta_n,
\]
where $ u \in \{ \pm 1 , \pm i, \pm j, \pm k \} $ is a unit of $\Lll$,
we can specify
that every $\beta_i$ except the first has $ a' = x $ without
losing any solutions. Thus the number of unique solutions
to equation~\ref{equation_4D} is at most
$ 384 \cdot 48^{n-1} $. Alternatively, we can factor out a unit
from the first factor $\beta_1$ to ensure that it has $ a' = x $, so the
resulting product is of the form $ u \beta_1 \beta_2 \cdots \beta_n $,
where each $ \beta_i $ belongs to
\[
T_2 = \{ x \pm b'i \pm c'j \pm d'k
\text{ : } \{ b', c', d' \} = \{ y, z, w \} \},
\]
with $ | T_2 | = 2^3 3! = 48 $.

If any two consecutive $ \beta $'s are conjugates, we can combine them
to form $ x^2 + y^2 + z^2 + w^2 $. As this is a real number, we can
factor it out. The product is now of the form
\[
(x^2 + y^2 + z^2 + w^2)^\gamma u \beta_1 \beta_2 \cdots \beta_{n-2\gamma},
\]
with $ \gamma $ ranging over $ 0, 1, \ldots, \left\lfloor \frac{n}{2} \right\rfloor $.

Assume $ n $ is even.
When $ \gamma = \frac{n}{2} $, the product is of the form
$ (x^2 + y^2 + z^2 + w^2)^{\frac{n}{2}} u  $, so there are 8 unique values, one for
each unit.
When $ \gamma = \frac{n}{2} - 1 $, the product is of the form
$ (x^2 + y^2 + z^2 + w^2)^{\frac{n}{2} - 1} u \beta_1 \beta_2  $. Although $ \beta_1 $
can be any element of $ T_2 $, $ \beta_2 $ cannot be equal to $ \overline{ \beta_1 } $,
so the number of solutions in this case is $ 8 \cdot 48 \cdot 47 $. The same restriction
applies for all $ \gamma < \frac{n}{2} $. For $ \gamma < \frac{n}{2} $,
the number of
solutions is $ 8 \cdot 48 \cdot 47^{n - 2\gamma - 1} $.
Summing over all possible values of $ \gamma $, the number of solutions is
\begin{align*}
8 + 8 \cdot 48 \cdot 47 + 8 \cdot 48 \cdot 47^3 + \ldots
+ 8 \cdot 48 \cdot 47^{n-1}
&= 8 + 8 \cdot 48 \cdot 47 \cdot \frac{47^{2 \cdot \frac{n}{2} } - 1 }{47^2 - 1}
\\
&= 8 + 8 \cdot 47 \cdot \frac{47^{n} - 1 }{47 - 1}
\\
&= 8 \left(1 +  \frac{47 (47^{n} - 1 ) }{46}\right)
\\
&= 8 \left( \frac{46 + (47^{n+1} - 47 ) }{46}\right)
\\
&= 8 \left( \frac{47^{n+1} -1 }{46}\right).
\end{align*}

Now assume $ n $ is odd.
When $ \gamma = \frac{n-1}{2} $, the product is of the form
$ (x^2 + y^2 + z^2 + w^2)^{\frac{n-1}{2}} u \beta_1  $, so there are $ 8 \cdot 48 $
unique solutions.
When $ \gamma = \frac{n-1}{2} - 1 = \frac{n-3}{2}  $, the product is of the form
$ (x^2 + y^2 + z^2 + w^2)^{\frac{n-3}{2}} u \beta_1 \beta_2 \beta_3  $. Although $ \beta_1 $
can be any element of $ T_2 $, $ \beta_2 \neq \overline{ \beta_1 } $
and $ \beta_3 \neq \overline{ \beta_2 } $,
so the number of solutions in this case is $ 8 \cdot 48 \cdot 47^2 $. The same restriction
applies for all $ \gamma < \frac{n-1}{2} $. For $ \gamma < \frac{n-1}{2} $,
the number of solutions is $ 8 \cdot 48 \cdot 47^{n - 2\gamma - 1 } $.
Summing over all values of $ \gamma $, the number of solutions is
\begin{align*}
8 \cdot 48 + 8 \cdot 48 \cdot 47^2 + 8 \cdot 48 \cdot 47^4 + \ldots
+ 8 \cdot 48 \cdot 47^{n-1}
&= 8 \cdot 48 \cdot \frac{47^{2 \cdot \frac{n+1}{2} } - 1 }{47^2 - 1}
\\
&= 8  \cdot \frac{47^{n+1} - 1 }{47 - 1}. \qedhere
\end{align*}
\end{proof}
\begin{remark}
Note that the proof of Theorem~\ref{theorem_counting_4D}
gives an upper bound for the number of solutions.
\end{remark}

\section{Conclusion}

\todo[inline]{Conclusion?}

\newpage

\newcommand{\journal}[6]{{\sc #1,} #2, {\it #3} {\bf #4} (#5), #6.}
\newcommand{\preprint}[3]{{\sc #1,} #2, preprint #3.}
\newcommand{\book}[4]{{\sc #1,} #2, #3, #4.}
\newcommand{\collection}[6]{{\sc #1,}  #2, #3, in {\it #4}, #5, #6.}
\newcommand{\JCTA}{J.\ Combin.\ Theory Ser.\ A}
\newcommand{\arxiv}[3]{{\sc #1,} #2, {\tt #3}.}
\newcommand{\article}[3]{{\sc #1,} #2, {\tt #3}.}
\newcommand{\journalfive}[5]{{\sc #1,} #2, {\it #3}  (#4), #5.}





\begin{thebibliography}{1}

%\bibitem{Hardy_and_Wright}
%\book{G.\ H.\ Hardy and E.\ M.\ Wright}
%         {An Introduction to the Theory of Numbers, 6th Edition}
%         {Oxford University Press, Oxford} 
%         {2008}



\bibitem{Davidoff_Sarnak_Valette}
\book{G.\ Davidoff, P.\ Sarnak, and A.\ Valette}
         {Elementary Number Theory,
           Group Theory,
           and Ramanujan Graphs}
         {Cambridge University Press}
         {2003}

\bibitem{Dickson}
\book{L.\ E.\ Dickson}
         {History of the Theory of Numbers}
         {AMS Chelsea Publishing}
         {1999}


%\bibitem{Jacobi}
%\journal{C.\ G.\ J.\ Jacobi}
%        {De compositione numerorum e quatuor quadratis}
%        {Journal f\"ur die reine und angewandte Mathematik}
%        {12}{1834}{167--172}


\bibitem{Ferrari}
\journal{F.\ Ferrari}
        {Risoluzione Dell'Equazione}
        {Supplemento al Periodico di Matematica}
        {11}{1908}{129--131}



\bibitem{Mordell}
\book{L.\ J.\ Mordell}
         {Diophantine Equations}
         {Academic Press, London and New York} 
         {1969}

\todo[inline]{Add a reference about quaternions, as well as my paper}
\end{thebibliography}






\end{document}



