%%%%%%%%%%%%%%%%%%%%%%%%%%%%%%%%
%
%
%  Pythagorean Quintuples and Quaternions
%  (journal article)
%  
%  by Theodore Ehrenborg
%
%
% 
% Last edited: September 1, 2019
%
%
%%%%%%%%%%%%%%%%%%%%%%%%%%%%%%%%%
%
%%%%%%%%%%%%%%%%%%%%%%%%%%%%%%%%%

\documentclass[14pt,table]{extarticle}

%%%%%%%%%%%%%%%%%%%%%%%%%%%%%%%%%
%
% pdf settings
%
%%%%%%%%%%%%%%%%%%%%%%%%%%%%%%%%%
%
\pdfpagewidth=8.5truein
\pdfpageheight=11truein





%
%%%%%%%%%%%%%%%%%%%%%%%%%%%%%%%%%




%\usepackage{anyfontsize}
\usepackage{amssymb, amsmath, fullpage, amsthm}
\usepackage{mathrsfs}
\usepackage{tikz}
\usepackage[title]{appendix}
\usepackage{mathrsfs}
\usepackage{gensymb}
\usepackage{enumerate}
\usepackage{caption}
\usepackage{comment}
\usepackage{xcolor}
\usepackage{diagbox}
\usepackage[colorinlistoftodos]{todonotes}
\usepackage{mwe}




\usetikzlibrary{math}
















\parskip2mm

\newtheorem{theorem}{Theorem}[section]
\newtheorem{hypothesis}[theorem]{Hypothesis}
\newtheorem{lemma}[theorem]{Lemma}
\newtheorem{proposition}[theorem]{Proposition}
\newtheorem{corollary}[theorem]{Corollary}
\newtheorem{conjecture}[theorem]{Conjecture}


\theoremstyle{definition}
\newtheorem{definition}[theorem]{Definition}
\newtheorem{example}[theorem]{Example}

\theoremstyle{remark}
\newtheorem{remark}[theorem]{Remark}
\newtheorem{remarks}[theorem]{Remarks}





\hyphenation{Hurwitz}

\font\german = eufm10 scaled\magstep1
\font\Cp = msbm10

\newcommand{\Ccc}{\mathbb C}
\newcommand{\Fff}{\mathbb F}
\newcommand{\Hhh}{\mathbb H}
\newcommand{\Nnn}{\mathbb N}
\newcommand{\Rrr}{\mathbb R}
\newcommand{\Sss}{\mathbb S}
\newcommand{\Zzz}{\mathbb Z}
\newcommand{\Lll}{\mathbb L}
\renewcommand{\Bbb}{\mathbb B}
\newcommand{\Ooo}{\mathbb O}
\newcommand{\Qqq}{\mathbb Q}


\newcommand{\vanish}[1]{}
\newcommand{\coveredby}{\prec}
\newcommand{\divides}{\mid}
\newcommand{\notdivides}{\nmid}
\newcommand{\timesdots}{\times \cdots \times}
\newcommand{\ascodd}{\asc_{\odd}}
\newcommand{\doubleprime}{\prime\prime}
\newcommand{\myfrac}[2]{#1 / #2}
\newcommand{\SSSS}{\mathfrak{S}}
\newcommand{\Gaussian}[2]{\genfrac{[}{]}{0pt}{}{#1}{#2}_q}
\newcommand{\fix}[1]{\todo[inline]{#1}}






\numberwithin{equation}{section}
%\usepackage[bindingoffset=0.2in,
%            left=1in,right=1in,top=1in,bottom=1in,
%            footskip=.25in]{geometry}%Sets margins
%\pagenumbering{gobble}%No page numbers




 


\DeclareMathOperator{\inv}{inv}
\DeclareMathOperator{\frst}{frst}
\DeclareMathOperator{\er}{er}
\DeclareMathOperator{\asc}{asc}
\DeclareMathOperator{\odd}{odd}
\DeclareMathOperator{\Imag}{Im}
\DeclareMathOperator{\Real}{Re}
\DeclareMathOperator{\N}{N}





\begin{document}
%\begin{landscape}


\title{Pythagorean Quintuples and Quaternions}


\author{\sc Theodore EHRENBORG\footnote{Clare College, Cambridge, {\tt theodore.ehrenborg@gmail.com}}
%\thanks{
%Corresponding author:
%Department of Mathematics,
%University of Kentucky,
%Lexington, KY 40506-0027,
%USA,
%{\tt theodore.ehrenborg@gmail.com}.}
%\:\: 
%\:\:
}



\date{\today}
%%%\date{Last edited on \today}

\maketitle



\begin{abstract}
This paper explores
the 
connections between quaternions and primitive Pythagorean quintuples. 
It is known that the square of a Gaussian integer
 is a Pythagorean triple $a^2 + b^2 = c^2$.
Less is 
known about the relationship between quaternions and Pythagorean quintuples
$a^2 + b^2 + c^2 + d^2 = e^2$.
We show that squaring a quaternion produces a subfamily
of Pythagorean quintuples.
Motivated by Conway and Smith's unique
factorization theorem for the
Hurwitz integers,
we present a more general version of
squaring a quaternion that
generates a larger subfamily of Pythagorean quintuples.
Using a counting argument and Jacobi's Four Square Theorem,
we show that unlike the characterization for Pythagorean triples,
the preceding characterization for Pythagorean quintuples is sparse.
Finally, we use a geometric approach to 
characterize 
all Pythagorean
quintuples.
We notice a similarity between the geometric approach
and the quaternion squaring approach in that they
differ by a geometrically defined constant.
\end{abstract}

\listoftodos


\section{Pythagorean Triples}




Recall nonnegative integers $a$, $b$ and $c$ form a
{\em Pythagorean triple} if
\[
    a^2 + b^2 = c^2.
\]
Furthermore, a {\em primitive Pythagorean triple}
is one where $\gcd(a, b,c) = 1$, that is,
$a$, $b$, and~$c$ have no common factor greater than $1$.
Pythagorean triples have the geometric interpretation that
they are the side lengths of a right triangle having integer
side lengths.




The following theorem characterizes primitive Pythagorean triples.
It was known by Diophantus~\cite[Page 93]{Heath} and
probably discovered by Euclid~\cite{Euclid}.
A proof can be found in Hardy and Wright's book;
see~\cite[XIII, 13.2]{Hardy_and_Wright}.



\vanish{
Also see page 295 of Heath for
Fermat's area of triangle result.}



\begin{theorem}[Euclid-Diophantus]
Given a primitive
Pythagorean triple $(a, b,c)$, there are integers
$x$ and $y$ such that $0 < y < x$, 
the integers
$x$ and $y$ are of opposite parity,
and
$\gcd(x, y) = 1$
so that
\begin{equation}
\label{equation_primitive_Pythagorean_triple}
     a = x^2-y^2, \:\:\:\: 
     b = 2xy, \mbox{  and  } \:\: c=x^2+y^2.
\end{equation}
\end{theorem}


\begin{example}
{\rm
For instance, 
for
$a = 5$, $b = 12$, and $c = 13$,
we have
$x = 3$
and $y = 2$.
Geometrically this gives the primitive Pythagorean triple
$(5, 12, 13)$ corresponding to the 
$5$-$12$-$13$ right triangle. 
}
\end{example}


\begin{comment}
\begin{figure}
\begin{center}
\begin{tabular}{ c|c|c|c|c|c } 

 $x$ & $y$ & $x^2 - y^2$ & $2xy$ & $x^2 - y^2$ & Check \\  
 \hline
 $1$ & $0$ & $1$ & $0$ & $1$ & $1^2 + 0^2 = 1^2$ \\  
 $2$ & $1$ & $3$ & $4$ & $5$ & $3^2 + 4^2 = 5^2$ \\  
 $3$ & $2$ & $5$ & $12$ & $13$ & $5^2 + 12^2 = 13^2$ \\  
 $3$ & $1$ & $8$ & $6$ & $10$ & $8^2 + 6^2 = 10^2$ \\  
 $4$ & $1$ & $15$ & $8$ & $17$ & $15^2 + 8^2 = 17^2$ \\  
 $4$ & $2$ & $12$ & $16$ & $20$ & $12^2 + 16^2 = 20^2$ \\  
 $4$ & $3$ & $7$ & $24$ & $25$ & $7^2 + 24^2 = 25^2$ \\  
\end{tabular}
\caption{Generating pairs for Pythagorean triples where
$0 \leq y < x \leq 4$.}
\end{center}
\label{fig:triples}
\end{figure}



\end{comment}




There is a well-known connection between primitive Pythagorean triples
and complex numbers. Consider any {\em Gaussian integer} $x+yi$,
where $x, y \in \Zzz$.
If $x + yi$ is squared, the real and imaginary
parts of the resulting Gaussian integer make up two legs of
a Pythagorean triangle as follows. 


\begin{theorem}
Any primitive Pythagorean triple $(a, b,c)$ with
$a^2 + b^2 = c^2$ can be represented
by the square of a Gaussian integer $x+yi$,
that is,
\[
    a =  \Re{(x+yi)^2},  \:\:\: b = \Im{(x+yi)^2},
    \mbox{   and   }
    c = |(x+yi)^2|,
\]
where $\Re$ and $\Im$ denote the real and imaginary parts of a complex number
and
$| \cdot |$ denotes the modulus of a complex number.
\end{theorem}



Of course, this fact can be proved with
algebra, but a more intuitive 
idea of a proof relies on the definition 
of multiplication. That is, the number $x+yi$ has a modulus, or     
distance from the origin, of $\sqrt{x^2+y^2}$. The modulus of the       
product of two complex numbers is the same as the product of
the moduli of the numbers.
So $(x+yi)^2$ has a modulus of $x^2+y^2$. Thus, the square of a 
Gaussian integer has integer coordinates and an integer distance
from the origin, giving a right triangle with integer sides.







\section{Pythagorean Quintuples via Squaring Quaternions}


Consider the system of quaternions, discovered by Sir William Hamilton 
in 1843~\cite{Hamilton}.
The quaternions can be thought
of as complex numbers extended to four dimensions.
The general
 form of a quaternion is $x+yi+zj+wk$,
where $x, y, z, w \in \Rrr$.
Up to a sign they have $4$ units: $1$, $i$, $j$, and $k$. Their most striking
property is that multiplication is not commutative. 
For example, $i \cdot j = k \neq j \cdot i = -k$.


\begin{definition}
The {\em modulus} 
of a quaternion
$q =  x+yi+zj+wk$
is
\[
   |q| = \sqrt{x^2 + y^2 + z^2 + w^2}.
\]
The {\em norm} of a quaternion is the modulus squared, that is,
\[
     N =  N(q) = x^2 + y^2 + z^2 + w^2.
\]
\end{definition}


It is straightforward to check the following property.
\begin{lemma}
\label{lemma_norm_multiplicative}
For two quaternions $q_1$ and $q_2$ the norm is multiplicative, that is,
\[
     N(q_1 \cdot q_2) = N(q_1) \cdot N(q_2).
\]
\end{lemma}



\begin{comment}


\begin{figure}
\begin{center}
\begin{tabular}{c||c|c|c|c}
$\cdot$ & $1$ & $i$ & $j$ & $k$ \\
\hline
\hline
$1$ & $1$ & $i$ & $j$ & $k$ \\
\hline
$i$ & $i$ & $-1$ & $k$ & $-j$\\
\hline
$j$ & $j$ & $-k$ & $-1$ & $i$ \\
\hline
$k$ & $k$ & $j$ & $-i$ & $-1$
\end{tabular}
\end{center}
\caption{Multiplication table for the quaternions.}
\label{fig:multiplication_table}
\end{figure}


\end{comment}





\begin{figure}
    \centering
    \begin{minipage}{0.45\textwidth}
        \centering
\begin{tikzpicture}

\tikzmath{
\scale = .7;
}

\draw[thick,->] (0,0) -- (\scale*8,0);
\draw[thick,->] (0,0) -- (0,\scale*4.5);
\draw[thick,->] (0,0) -- (\scale*-2.5,0);
\draw[thick,->] (0,0) -- (0,\scale*-2.5);
\draw[dashed] (\scale*6,0) -- (\scale*6,\scale*2);
\fill (\scale*6,\scale*2) circle (\scale*.1cm);
\draw (0,0) -- (\scale*6,\scale*2) node[anchor=south west] {$x+yi$};
\draw (\scale*4.5,\scale*1.5) node[anchor=south east] { $\sqrt{x^2 + y^2}$};
\draw (\scale*6,\scale*1) node[anchor=west] {$y$};
\draw (\scale*3,0) node[anchor=north] {$x$};
\end{tikzpicture}
\caption{Geometric representation of a Gaussian integer.}
\label{fig:Pythagorean}
    \end{minipage}\hfill
    \begin{minipage}{0.45\textwidth}
        \centering
\begin{tikzpicture}


\tikzmath{
\scale = .7;
}
\draw[thick,->] (0,0) -- (\scale*8,0);
\draw[thick,->] (0,0) -- (0,\scale*4.5);
\draw[thick,->] (0,0) -- (\scale*-2.5,0);
\draw[thick,->] (0,0) -- (0,\scale*-2.5);
\draw[dashed] (\scale*5,0) -- (\scale*5,\scale*4);
\fill (\scale*5,\scale*4) circle (\scale*.1cm);
\draw (0,0) -- (\scale*5,\scale*4) node[anchor=south west] {$(x+yi)^2$};
\draw (\scale*3.5,\scale*2.6) node[anchor=south east ] {$x^2 + y^2$};
\draw (\scale*5,\scale*2) node[anchor=west] {$2xy$};
\draw (\scale*2.5,0) node[anchor=north] {$x^2-y^2$};


\end{tikzpicture}
\caption{A Gaussian integer $x+yi$ after squaring.}
\label{fig:Pythagorean_squared}
    \end{minipage}
  \end{figure}




\begin{theorem}
\label{thm:as_squares}
There are Pythagorean quintuples 
$a^2 + b^2 +c^2 +d^2 = e^2$ which can be represented
by the square of a quaternion $x+yi+zj+wk$ where $x, y, z, w \in \Zzz$.
The form of these Pythagorean quintuples is
\[
     (a,b,c,d,e) = (2xy, 2xz, 2xw, x^2 - y^2 - z^2 - w^2, x^2 + y^2 + z^2 +w^2).
\]
\end{theorem}
\begin{proof}
The general
 form of a quaternion is $q = x+yi+zj+wk$.
Squaring this gives
\begin{eqnarray*}
     q^2 = (x + yi + zj +wk)^2
      &=& (x^2- y^2 -z^2 - w^2)+
          (2xy)i + (2xz)j + (2xw)k.
\end{eqnarray*}
The
norm of $q$ squared is
\begin{align*}
     N(q^2) &=  (x^2 + y^2 + z^2 +w^2)^2 \\
      &= (x^2- y^2 -z^2 - w^2)^2 +
          (2xy)^2 + (2xz)^2 + (2xw)^2.
\end{align*}
This means that 
$(2xy, 2xz, 2xw, x^2 - y^2 - z^2 - w^2, x^2 + y^2 + z^2 +w^2)$
is a Pythagorean quintuple.
%This is the same formula as the one found geometrically.
\end{proof}

\vanish{
\begin{eqnarray*}
     (ix+jy+kz+w)^2
      &=& -x^2+kxy-jxz+ixw-kxy-y^2+iyz+jyw\\
      & & + jxz-iyz-z^2+kzw+ixw+jyw+kzw+w^2\\
      &=& i(2xw)+j(2yw)+k(2zw)-(x^2+y^2+z^2-w^2)
\end{eqnarray*}
}

\noindent

Observe 
that
$1^2+ 1^2+ 3^2+ 5^2= 6^2$ is a 
primitive Pythagorean quintuple, but none of the
legs  $1$, $1$, $3$ and $5$ are even.
Thus Theorem~\ref{thm:as_squares} is a partial characterization
of Pythagorean quintuples.




\section{Pythagorean Quintuples via Multiplying two Hurwitz Integers}




John Conway and Derek Smith~\cite{Conway_and_Smith}
have investigated the relationship of quaternions
to symmetry groups.  See the review by Baez~\cite{Baez}
for a synopsis of their work.
Conway and Smith's unique factorization theorem 
(see Theorem~\ref{thm:unique_factorization}) suggests a way to extend
the relationship between Pythagorean quintuples
and quaternions.


In order to do this, we recall the Hurwitz integers.

\begin{definition}
A quaternion $q = x+yi+zj+wk$ is a
{\em Hurwitz integer}
if either $x, y, z, w \in \Zzz$ 
or
$x, y, z, w \in \Zzz + \frac{1}{2}$,
where
$\Zzz + \frac{1}{2}$ denotes the half integers.
In the case that
$x, y, z, w \in \Zzz$
we say $q$ is a {\em Lipschitz integer}.
\end{definition}
Let $\Hhh$ denote the set of Hurwitz integers.
Because the Hurwitz integers form a well-packed lattice,
they are suitable for error-correcting codes.
See the recent paper of G\"uzeltepe~\cite{Guzeltepe}.
Conway and Smith found that the Hurwitz integers satisfy
the following unique factorization 
property~\cite[Ch.\ 5, Theorem 2]{Conway_and_Smith}.
To do this we need one more definition.



\begin{definition}
A Hurwitz integer $q \in \Hhh$ is
{\em primitive} if there is no natural number $n > 1$ that divides it.
\end{definition}



\begin{theorem}[Conway-Smith]
\label{thm:unique_factorization}
Let $q \in \Hhh$ be a primitive Hurwitz integer with norm~$N$.  Suppose $N = p_1 p_2 \cdots p_k$ where
$p_1, p_2, \ldots, p_k$ are distinct prime numbers.
Then
$q$ can be factored as a product of Hurwitz integers
\[
   q = P_1 \cdot P_2 \cdots P_k,
\]
where the norm of $P_i$ is
given by $N(P_i) = p_i$ for $i = 1, \ldots, k$.
This factorization is unique up to unit migration, that is,
all other factorizations corresponding to
the product  $N = p_1 p_2 \cdots p_k$ 
are of the form
\[
   q = (P_1 U_1) \cdot (U_1^{-1} P_2 U_2) \cdots (U_{k-1}^{-1}P_k),
\]
where the $U_i$ are Hurwitz units, that is,
Hurwitz integers with norm $1$.
\end{theorem}
The $24$ Hurwitz units are:
\[
  \pm 1, \pm i, \pm j, \pm k, \frac{1}{2}(\pm 1 \pm i \pm j \pm k).
\]


Just as the square of a Gaussian integer must have
an integer modulus, the square of a Hurwitz integer has an integer modulus
and therefore a norm that is a perfect square. 
As a result of Theorem~\ref{thm:unique_factorization},
a quaternion
built by using the coefficients of a Pythagorean quintuple can 
always be factored into two 
quaternions with the same norm.
\fix{I can also deduce the last sentence from Sarnak et al.'s book}

What does this mean in concrete terms? We give an example.

\begin{example}
{\rm The quintuple 
\[
     1^2+ 1^2+ 3^2+ 5^2= 6^2
\]
cannot be represented by the squaring method
of Theorem~\ref{thm:as_squares},
but it can be translated into 
a quaternion
\[
      1 + i + 3j + 5k.
\]
We can factor out a $2$, leaving us with the
primitive Hurwitz integer
\[
     h= \frac{1}{2} + \frac{1}{2}i + \frac{3}{2}j + \frac{5}{2}k.
\]
Note the norm of $h$ is $N(h) = 9 = 3 \cdot 3$.
Since 
\[
\frac{1}{2} + \frac{1}{2}i + \frac{3}{2}j + \frac{5}{2}k
=
\left(\frac{1}{2}+\frac{1}{2}i+\frac{3}{2}j-\frac{1}{2}k\right)
\left(\frac{1}{2}-\frac{3}{2}i+\frac{1}{2}j+\frac{1}{2}k\right) 
\]
then 
the four half integers $\frac{1}{2}$, $\frac{1}{2}$, $\frac{1}{2}$, 
and $\frac{3}{2}$ 
generate $1^2+ 1^2+ 3^2+ 5^2= 6^2$. 
}
\end{example}

What is nice in this example is that the two Hurwitz integers 
$P_1$ and $P_2$ whose
product gives the primitive Hurwitz integer $h$ have the property that
their coefficients, in absolute value and reordered, coincide.

\begin{theorem}
\label{thm:as_products}
There are Pythagorean quintuples 
$a^2 + b^2 +c^2 +d^2 = e^2$ that can be represented
by the product of two Hurwitz integers
$x + yi + zj + wk$ and
$x' + y'i + z'j + w'k$
where
the elements
$|x|, |y|, |z|, |w|$ correspond
to some ordering of the elements
$|x'|, |y'|, |z'|, |w'|$.
%The 57 forms of these Pythagorean quintuples 
%appear in Appendix~\ref{appendix_C}.
\end{theorem}
\begin{proof}
Consider the product of two Hurwitz integers
$P_1 = x + yi + zj + wk$ and
$P_2 = x' + y'i + z'j + w'k$.  
Since the second Hurwitz integer $P_2$ 
is just a reordering of
the coefficients $x$, $y$, $z$ and $w$ of $P_1$, 
with possible sign changes,
the norms of these two Hurwitz integers
coincide, that is,
$N(P_1) = N(P_2) = \varepsilon$.


We claim $\varepsilon \in \Zzz$.  If $P_1$ (and hence $P_2$)
is a Lipschitz integer then 
the norm $\varepsilon$ is obviously an integer.
Otherwise, write 
\[
  P_1 = \frac{\alpha}{2} + \frac{\beta}{2} i + 
        \frac{\gamma}{2} j + \frac{\delta}{2} k,
\]
where
$\alpha$, $\beta$, $\gamma$ and $\delta$ are all odd integers.
The norm of $P_1$ is then
\[
     N(P_1) = \frac{1}{4} \cdot 
              (\alpha^2 + \beta^2 + \gamma^2 + \delta^2).
\]
Since
\[
  \alpha^2 \equiv \beta^2 \equiv \gamma^2 \equiv \delta^2 \equiv 1 \mod 4,
\]
their sum is divisible by $4$.
Hence $N(P_1) = \varepsilon \in \Zzz$.
By Lemma~\ref{lemma_norm_multiplicative},
$N(P_1 \cdot P_2) = \varepsilon^2$
and 
$|P_1 \cdot P_2| = \varepsilon$.
Since $|P_1 \cdot P_2|$ is an integer and 
$P_1 \cdot P_2 = a + bi + cj + dk$ has
coefficients that are either integers or half-integers
then
either
$(a, b,c, d,e)$ is a Pythagorean quintuple
or
both
$(a, b,c, d,e)$ and
$(2a,2b,2c,2d,2e)$ are Pythagorean quintuples,
where
$e = \varepsilon$.
\end{proof}





One can check that 
$1^2+ 2^2+ 8^2+ 10^2  = 13^2$
is a primitive Pythagorean quintuple that
cannot be written as the product of two quaternions
using the same coefficients.
For example
\[
  -2 + 10 i + j + 8k = (2 + 3i) \cdot (2 + 2i + 2j + k).
\]
See Appendix~\ref{appendix_A} for more details.
Thus Theorem~\ref{thm:as_products} 
is a partial characterization of Pythagorean quintuples 
which arise from products of two related Hurwitz integers.


\vanish{
Moreover, 103 out of 337 quintuples were unable to be 
represented with Hypothesis \ref{as_products}. Still, this is an
 improvement over Hypothesis \ref{as_squares}, which could not
 represent 259 of the quintuples. 
}





\section{An Asymptotic Analysis of Theorem~\ref{thm:as_products}}



In this section we analyze the general question
of finding primitive Pythagorean quintuples of magnitude $\epsilon$ using
a polynomial parametrization in four parameters.



Let us review some definitions. 
We say a  {\em generating quadruple} is a list of
either four integers or four half-integers.
The {\em magnitude} of a Pythagorean 
quintuple is the value of its largest element, that is,
the magnitude of $a^2 + b^2 + c^2 + d^2 = e^2$
is $e$.
For instance, the magnitude of 
$0^2+ 0^2+ 8^2+ 15^2  = 17^2$ is $17$. 


\vanish{
A formula is a function that produces a Pythagorean quintuple when
given a generating quadruple. Recall that the norm of the generating
quadruple equals the magnitude of the Pythagorean quintuple that the
quadruple generates, no matter the formula used. As a result, the
representation found in Theorem \ref{like_Hatcher} does not qualify as
a formula because dividing by the greatest common factor may change
the magnitude.  The intent of this paper was to find a
parametrization for primitive Pythagorean quintuples: to find a list
of formulas whose combined output was every primitive Pythagorean
quintuple.  }



This section will require the Jacobi Four Square Theorem, proved by
Jacobi in 1834~\cite{Jacobi}.  This is a counting
version of Lagrange's theorem that every integer $n$ can
be written as a sum of four squares.
A proof using Hurwitz integers appears in
Hardy and Wright~\cite[Chapter XX, Theorem 386]{Hardy_and_Wright}.
\cite{Lalin} expresses this theorem differently.
\begin{theorem}[Jacobi]
\label{thm:representations}
The number of solutions to $x_1^2 +  x_2^2 +  x_3^2 +  x_4^2  = n$ is 
\[
8\sum_{d|n,4\nmid d}{d},
\]
where
the order and sign of a solution $(x_1,x_2,x_3,x_4)$ matter. 
\end{theorem}


\begin{theorem}
\label{thm:analysis}
For any prime $\epsilon >> 0$, the number of primitive Pythagorean 
quintuples of magnitude $\epsilon$ is greater than the number of 
Pythagorean quintuples of magnitude $\epsilon$ that can be generated using 
any parametrization in four variables $x_1, x_2, x_3, x_4$ that
obeys the following conditions:
\begin{itemize}
\item[(i)]
The parametrization always 
generates Pythagorean quintuples.

\item[(ii)]
Their squares give the sum
\[
     x_1^2 + x_2^2 + x_3^2 + x_4^2 = \epsilon
\]

\item[(iii)]
The values of $x_1, x_2, x_3, x_4$ are either all integers or
all half-integers.
\end{itemize}
\end{theorem}


Observe that the 57 representations found as a consequence of
Theorem~\ref{thm:as_products} are such parametrizations.


We now give a 
proof of Theorem~\ref{thm:analysis}.

\begin{proof}
Let $\epsilon$ be a large prime. 
By Lagrange's Theorem every integer can be written as the sum of 
four squares, 
so
there must be at least one solution to
$\alpha^2 + \beta^2 + \gamma^2 + \delta^2 = \epsilon^2$, where $\alpha$, $\beta$, $\gamma$, and $\delta$ are
 nonnegative integers. This is a Pythagorean quintuple of magnitude $\epsilon$.


We are interested in the number of Pythagorean quintuples of magnitude
$\epsilon$
where the
order of $\alpha$, $\beta$, $\gamma$, and $\delta$ does not matter.  
Since $\epsilon$ is a prime,
the integers $\alpha$, $\beta$, $\gamma$, and $\delta$ share no
common factor other than $1$ or $\epsilon$. 
Thus, $\alpha^2 + \beta^2 + \gamma^2 + \delta^2 = \epsilon^2$ is a primitive Pythagorean
quintuple. 


Let $\psi$ be the number of primitive
Pythagorean quintuples of magnitude $\epsilon$. Note that $\psi$ does not count
Pythagorean quintuples whose addends  are squares of  half-integers.
Let $\lambda$ be the number of solutions to 
$x_1^2 +  x_2^2 +  x_3^2 +  x_4^2 = \epsilon^2$
where order and sign matter.
The only divisors of $\epsilon^2$ are
$\epsilon^2$, $\epsilon$, and $1$. Using 
Jacobi's Four Square Theorem~\ref{thm:representations}, 
\[
     \lambda = 8(1 + \epsilon + \epsilon^2).
\]
However, we are interested in the number of Pythagorean quintuples
where the order and sign do not matter. Since $\lambda$ counts at least one
solution where one of $x_1,x_2,x_3,x_4$ is negative, $\lambda > \psi$.



We next look for a lower bound for $\psi$.
In the case that a Pythagorean quintuple counted by $\psi$ 
has distinct and positive $\alpha,\beta,\gamma,\delta$,
Jacobi's theorem vastly  overestimates the number of
Pythagorean quintuples. 
Such a Pythagorean quintuple can have its terms  permuted in 
$4! = 24$ ways and can have $2^4 = 16$ different arrangements of sign.


For example, the Pythagorean quintuple
 $1^2 + 2^2 + 10^2 + 16^2 = 19^2$ is overcounted by
Jacobi's theorem as
$(1,2,10,16)$ , $(2,1,-16,10)$, $(-16,1,-2,10)$, etc.
All are valid solutions to
$x_1^2 +  x_2^2 +  x_3^2 +  x_4^2 = 19^2$.
Thus, such a Pythagorean quintuple
is overcounted by $24 \cdot 16 = 384$ times in $\lambda$. 
This is the maximum amount that such a Pythagorean
quintuple could be overcounted.
Thus,
\[
     \psi \geq \frac{\lambda}{384}.
\] 
Since we know the value of $\lambda$ when $\epsilon$ is prime,
\[
     \psi \geq \frac{ 8(1+\epsilon+\epsilon^2) }{384}.
\] 

In the case of
finding parametrizations for the primitive Pythagorean
quintuples whose addends are all squares of half-integers,
let $\kappa$ be the total  number of 
primitive Pythagorean quintuples of magnitude $\epsilon$. Since $\kappa$ is clearly 
at least $\psi$, we know that 
\begin{equation}
\label{lower}
\kappa \geq \frac{ 8(1+\epsilon+\epsilon^2) }{384}
\end{equation}

Let us shift our focus to the number of generating quadruples that could generate a Pythagorean quintuple of magnitude $\epsilon$.
The quadruple has the form $(a,b,c,d)$, 
where $a^2 + b^2 + c^2 + d^2 = \epsilon$. 
Here $a$, $b$, $c$, and $d$ are all integers or all half-integers.

Since $a^2 + b^2 + c^2 + d^2 = \epsilon$, we multiply by $4$ to obtain
\begin{align*}
     & 4a^2 + 4b^2 + 4c^2 + 4d^2 = 4\epsilon\\
     & (2a)^2 + (2b)^2 + (2c)^2 + (2d)^2 = 4\epsilon
\end{align*}
Let $m_1 = 2a$, $m_2 = 2b$, $m_3 = 2c$, 
and $m_4 = 2d$.
Note $m_i \in \Zzz$ for $i = 1, \ldots, 4$ and
\begin{equation}
\label{quadruple_equation}
m_1^2 + m_2^2 + m_3^2 + m_4^2 = 4\epsilon
\end{equation}
Note that this gives 
a simple bijection between a generating quadruple
 $(a, b, c, d)$ and a quadruple of four integers $(m_1,m_2,m_3,m_4)$ whose norm is $4\epsilon$.

We again refer to
Jacobi's Theorem to
determine how many solutions
\eqref{quadruple_equation} has.
Since $\epsilon$ is a large
prime,
$4\epsilon$ is divisible by $1$, $2$, $4$, $\epsilon$, $2\epsilon$, and $4\epsilon$. 
Leaving 
out the divisors that are divisible by $4$, 
the number of solutions to \eqref{quadruple_equation}
is 
\[
     8(1 + 2 + \epsilon + 2\epsilon) = 8(3 + 3\epsilon) = 24(1 + \epsilon).
\]


Assume that there are $\tau$ parametrizations
that meet the three conditions in the theorem. 
Let $\omega$ be the number of Pythagorean quintuples produced by applying
the $\tau$ parametrizations to the $24(1+\epsilon)$ generating quadruples. 
We have
an upper bound for $\omega$.
\begin{equation}
\label{upper}
\omega \leq 24\tau(1+\epsilon)
\end{equation}



Observe that in \eqref{lower} $\kappa$ grows with the square of $\epsilon$, but in 
\eqref{upper} $\omega$ grows linearly  in $\epsilon$.

When $\epsilon$ is a sufficiently large prime,
\begin{align*}
   & \frac{ 8(1+\epsilon+\epsilon^2) }{384} > 24\tau(1+\epsilon) \\
   & \kappa \geq \frac{ 8(1+\epsilon+\epsilon^2) }{384} > 24\tau(1+\epsilon) \geq \omega \\
   & \kappa > \omega
\end{align*}
where in the second inequality we used
\eqref{lower} and \eqref{upper}.
Thus, 
for any prime $\epsilon$ above a certain bound, 
the number of primitive  Pythagorean 
quintuples of magnitude $\epsilon$ is greater than the number of 
Pythagorean quintuples of magnitude $\epsilon$ that can be generated 
using parametrizations of this type.
Since there are an infinite number of primes larger 
than that bound, there is an 
infinite number of primitive  Pythagorean 
quintuples that cannot be generated by
 any finite list of such parametrizations.
\end{proof}



Note that this result also excludes the possibility of 
using the Hurwitz units as multipliers to generate
all primitive Pythagorean quintuples. An infinite
number would still evade representation. 




\section{Characterization of Pythagorean Quintuples:  
A Geometric Approach}

\fix{This was done algebraically in Mordell.}

In this section we derive
an expression 
that characterizes {\em all} Pythagorean quintuples.


\begin{theorem}
\label{thm:like_Hatcher}
All primitive Pythagorean quintuples can
be written in the form
\[(
2ad
,2bd
,2cd
,a^2+b^2+c^2-d^2
,a^2+b^2+c^2+d^2)/k,
\]
where $k = \gcd(2ad,2bd,2cd,a^2 + b^2 + c^2 -d^2)$
and $a, b, c, d \in \Zzz$.
\end{theorem}
\begin{proof}
Consider the $3$-dimensional
sphere $x^2 + y^2 + z^2 +w^2 =1$
in $\Rrr^{4}$,
denoted by $\Sss^3$.
Let $\alpha^2 +\beta^2 +\gamma^2 + \delta^2 = \epsilon^2$ be a primitive Pythagorean quintuple.


Since 
\[
     \left(\frac{\alpha}{\epsilon}\right)^2 +\left(\frac{\beta}{\epsilon}\right)^2+
     \left(\frac{\gamma}{\epsilon}\right)^2+\left(\frac{\delta}{\epsilon}\right)^2 = 1,
\]
the point $(\frac{\alpha}{\epsilon},\frac{\beta}{\epsilon},\frac{\gamma}{\epsilon},\frac{\delta}{\epsilon})$ 
is on the sphere $\Sss^3$. 
Note that there exists a bijection between rational 
points on $\Sss^3$ and primitive Pythagorean quintuples.
If a line is drawn between the points $(0,0,0,1)$ and
$(\frac{\alpha}{\epsilon},\frac{\beta}{\epsilon},\frac{\gamma}{\epsilon},\frac{\delta}{\epsilon})$,
it has rational slope.
Furthermore, 
this line passes through the three-dimensional 
subspace $w=0$ at the point $(m,n,p,0)$,
where 
$m$, $n$, and $p$ are positive rational numbers. 
Notice that the line through $(0,0,0,1)$ gives a
bijection between rational
points on $\Sss^3$ and rational points on $w=0$.


The line satisfies the following conditions:
\[
     \frac{\partial y}{\partial x} =\frac{n}{m} ,\:\:\: 
     \frac{\partial z}{\partial x}=\frac{p}{m}, \:\:\:
     \frac{\partial w}{\partial x}=-\frac{1}{m}.
\]
Therefore, the line satisfies these conditions as well:
\[
      y=\frac{n}{m}x , \:\:\: z=\frac{p}{m}x, \:\:\:w=-\frac{1}{m}x+1.
\]

We use these conditions to solve for the intersection
 of the line and the sphere $\Sss^3$, which 
we already know are the two points $(0,0,0,1)$ and 
 $(\frac{\alpha}{\epsilon},\frac{\beta}{\epsilon},\frac{\gamma}{\epsilon},\frac{\delta}{\epsilon})$.


We have
\begin{align*}
     & x^2 + y^2 + z^2 +w^2 =1
\\
     & x^2 + \left(\frac{n}{m}x\right)^2 + 
        \left(\frac{p}{m}x\right)^2 
        +\left(-\frac{1}{m}x+1\right)^2 = 1
\\
%     & x^2 + \frac{n^2}{m^2}x^2 + \frac{p^2}{m^2}x^2
%        +\frac{1}{m^2}x^2-\frac{2}{m}x+1=1
%\\
%     & x^2\left(1 + \frac{n^2}{m^2} + \frac{p^2}{m^2}
%       +\frac{1}{m^2}\right)-\frac{2}{m}x = 0
%\\
     & x\left(x\left(1 + \frac{n^2}{m^2} + \frac{p^2}{m^2} 
        +\frac{1}{m^2}\right)-\frac{2}{m}\right) = 0
\end{align*}
The solution $x=0$ refers to the point $(0,0,0,1)$.
We continue solving for the other intersection point.
\begin{align*}
     & x\left(1 + \frac{n^2}{m^2} + \frac{p^2}{m^2} 
      +\frac{1}{m^2}\right)-\frac{2}{m}=0     
%\\
%     & x\left(\frac{m^2 + n^2 + p^2 +1}{m^2}\right)-\frac{2}{m}=0
%\\
%     & x\left(\frac{m^2 + n^2 + p^2 +1}{m^2}\right)=\frac{2}{m}
\\
     & x = \frac{2m}{m^2 + n^2 + p^2 +1}     
\end{align*}



This solution is the x-coordinate of the point 
$\left(\frac{\alpha}{\epsilon},\frac{\beta}{\epsilon},\frac{\gamma}{\epsilon},\frac{\delta}{\epsilon}\right)$.
We solve for the other coordinates of this point.


\[y=\frac{n}{m}x=\frac{n}{m}\cdot\frac{2m}{m^2 + n^2 + p^2 +1}
=\frac{2n}{m^2 + n^2 + p^2 +1}\]

\[z=\frac{p}{m}x=\frac{p}{m}\cdot\frac{2m}
{m^2 + n^2 + p^2 +1}=\frac{2p}{m^2 + n^2 + p^2 +1}\]

\begin{align*}
    w = -\frac{1}{m}x+1 &= -\frac{1}{m}\cdot\frac{2m}{m^2 + n^2 + p^2 +1}+1
\\
%      & = \frac{-2}{m^2 + n^2 + p^2 +1}
%       + \frac{m^2 + n^2 + p^2 +1}{m^2 + n^2 + p^2 +1}
%\\
      &= \frac{m^2 + n^2 + p^2 -1}{m^2 + n^2 + p^2 +1}.
\end{align*}

Since $m$, $n$, and $p$ are rational, we can write the following:
\[
     m=\frac{a}{d}, \:\:
     n=\frac{b}{d}, \:\:
     p=\frac{c}{d},
\]
where $a$, $b$, and $c$ 
are non-negative integers and $d$ is a positive integer.



Rewriting the coordinates gives:
\[x=\frac{2m}{m^2 + n^2 + p^2 +1}
=\frac{2 \cdot \frac{a}{d}}{\left(\frac{a}{d}\right)^2+\left(\frac{b}{d}\right)^2+\left(\frac{c}{d}\right)^2+1}
=\frac{2ad}{a^2+b^2+c^2+d^2}\]

\[y=\frac{2n}{m^2 + n^2 + p^2 +1}                    
=\frac{2\cdot\frac{b}{d}}{\left(\frac{a}{d}\right)^2+\left(\frac{b}{d}\right)^2+\left(\frac{c}{d}\right)^2+1} 
=\frac{2bd}{a^2+b^2+c^2+d^2}\]

\[z=\frac{2p}{m^2 + n^2 + p^2 +1}                                               
=\frac{2\cdot\frac{c}{d}}{\left(\frac{a}{d}\right)^2+\left(\frac{b}{d}\right)^2+\left(\frac{c}{d}\right)^2+1} 
=\frac{2cd}{a^2+b^2+c^2+d^2}\]


\[w=\frac{m^2 + n^2 + p^2 -1}{m^2 + n^2 + p^2 +1}
=\frac{\left(\frac{a}{d}\right)^2+\left(\frac{b}{d}\right)^2+\left(\frac{c}{d}\right)^2-1} 
{\left(\frac{a}{d}\right)^2+\left(\frac{b}{d}\right)^2+\left(\frac{c}{d}\right)^2+1} 
=\frac{a^2+b^2+c^2-d^2}{a^2+b^2+c^2+d^2}\]






\begin{figure}
\begin{center}  
\begin{tikzpicture}


\draw[thick,->] (0,0) -- (8,0); 
\draw[thick,->] (0,0) -- (0,4.5);
\draw[thick,->] (0,0) -- (-4.5,0); 
\draw[thick,->] (0,0) -- (0,-4.5);
\draw (0,0) circle (3cm);
\fill (0,3) circle (.1cm);
\fill (6,0) circle (.1cm);
\fill (2.4 , 1.8) node[anchor=south west] 
{$(\frac{\alpha}{\gamma},\frac{\beta}{\gamma})$} circle (.1cm);
\draw (0,3) node[anchor=south east] {$(0,1)$};
\draw (0,3) -- (6,0) node[anchor=south west] {$(p,0)$};



\end{tikzpicture}
\end{center}  
\caption{A two-dimensional version of the geometric argument
to characterize Pythagorean triples 
$\alpha^2~+~\beta^2~=~\gamma^2$}
\fix{method?}
\end{figure}





This particular $(x,y,z,w)$ is the same point as
$\left(\frac{\alpha}{\epsilon},\frac{\beta}{\epsilon},\frac{\gamma}{\epsilon},\frac{\delta}{\epsilon}\right)$. 
Setting these equal to each other gives:

\[x=\frac{2ad}{a^2+b^2+c^2+d^2}=\frac{\alpha}{\epsilon}\]

\[y=\frac{2bd}{a^2+b^2+c^2+d^2}=\frac{\beta}{\epsilon}\]

\[z=\frac{2cd}{a^2+b^2+c^2+d^2}=\frac{\gamma}{\epsilon}\]

\[w=\frac{a^2+b^2+c^2-d^2}{a^2+b^2+c^2+d^2}=\frac{\delta}{\epsilon}.\]
Therefore, we can write
\begin{align*}
     & k\alpha = 2ad     
\\
     & k\beta = 2bd     
\\
     & k\gamma = 2cd     
\\
     & k\delta = a^2 + b^2 + c^2 - d^2     
\\
     & k\epsilon = a^2 + b^2 + c^2 + d^2
\end{align*}     
\noindent
where $k$ is a positive rational number. 
Write $k$ as $\frac{t}{u}$, 
where $t$ and $u$ are positive integers with no common factor 
greater than $1$. 
Thus
$\frac{\alpha}{u}$, $\frac{\beta}{u}$, $\frac{\gamma}{u}$,
$\frac{\delta}{u}$, and $\frac{\epsilon}{u}$ are all positive integers.
 Therefore, $u$ is a common factor of $\alpha$, $\beta$, $\gamma$, $\delta$, and $\epsilon$. 
Since this Pythagorean quintuple is primitive, $u$ must be $1$.
%Is it necessary to prove this?
Therefore, $k$ is a positive integer.
\end{proof}








\begin{remarks}
{\rm
We end with two comments.
First, the approach we give to prove
Theorem~\ref{thm:like_Hatcher}
is similar to the geometric proof given in
Hatcher's notes~\cite{Hatcher}
for Pythagorean triples and
quadruples.
However, it does not appear in the literature.
Secondly,
the expression in Theorem~\ref{thm:like_Hatcher}
looks similar to the square of a quaternion
from the proof of Theorem~\ref{thm:as_squares}, namely
\[
     q^2 = (x + yi + zj +wk)^2
      = (x^2- y^2 -z^2 - w^2)+
          (2xy)i + (2xz)j + (2xw)k.
\]
The difference is in Theorem~\ref{thm:like_Hatcher} we
divide by the greatest common divisor of the four terms.
This projects
Pythagorean quintuples onto primitive Pythagorean quintuples. 
}
\end{remarks}


\section{Conclusion}



The original
goal of this project was to find a parametrization, preferably a polynomial one,  of 
Pythagorean~quintuples in four parameters.
Such a parametrization of quintuples
has been done using $12$ parameters in~\cite{Frisch_and_Vaserstein},
but the method involved a
specific case of sextuples, not a direct relationship between 
quintuples and quaternions. 
This project has shown that a
direct relationship is not trivial.



There are several paths
 for future research:
\begin{enumerate}

%Not true
%\item
%{ The Hurwitz integers have
% 24 units, so it may be possible to represent any Pythagorean 
%quintuple as a product of two quaternions and a unit. 
%}

%
%\item 
%{ Why can almost every 
%quintuple be generated
%in a multiple of 12 times? I suspect that this is an artifact
%of my program. 
%}


\item
{Conway and Smith have developed connections between the
 quaternions and 
orthogonal groups.
Their ideas may provide an important
 geometrical interpretation. 
}


\item
{ Perhaps the obvious geometrical analogue of Pythagorean quintuples
  --- the volumes of the facets of what we will call a
  tetrarectangular 4-dimensional simplex --- has a deeper meaning.  }


%\item
%{ Perhaps the obvious geometrical
%analogue of Pythagorean quintuples --- sections of a 4-dimensional
% rectangular prism, just as a right triangle is a section of
% a rectangle --- has deeper meaning.   
%}

\item

{In the 1930s, B.\ Berggren \cite{Berggren} discovered a set of
transformations that iteratively generate all primitive Pythagorean
triples.  F.\ J.\ M.\ Barning \cite{Barning} independently found
this method and reformulated it into a set of matrices, and Conrad 
\cite{Conrad} provided a geometric interpretation.  Notably,
these transformations can be shown as the Barning-Hall tree, which
contains all primitive Pythagorean triples.  This approach is 
completely different from the methods in this paper.}

%Berggren was not just republishing a result by H. Rath.  Rath appears
%to have compiled every known result on Pythagorean triples and not
%cited his sources.  https://hdl.handle.net/2027/mdp.39015036980533

%This paper seems like real math, but is not related enough to my
%project. It has a correct overview of the topic: A Discussion on
%Primitive Pythagorean Triples and Primitive Pythagorean Primes
%Duvvuri Surya Rahul and Snehanshu Saha


\end{enumerate}

The author thanks Peter Sarnak for a discussion in summer 2018.

This project was presented at the 2018 Intel International Science and Engineering Fair.


\begin{appendices}
%%\appendix

\section{Detailed Analysis of a Pythagorean Quintuple That Does Not Arise from
Theorem~\ref{thm:as_products}}
\label{appendix_A}



Consider the Pythagorean quintuple
$1^2 + 2^2 + 8^2 + 10^2 = 13^2$.
It
cannot be represented as in
Theorem~\ref{thm:as_products}.
If it could, it would use one of the following generating lists.
The
coefficients are the only Hurwitz integers to have a modulus of 13:

\fix{Get rid of the words generating list, generating quadruple, etc.}

\begin{enumerate}

\item $ \frac{7}{2},\frac{1}{2}, \frac{1}{2}, \frac{1}{2}  $\\
How could this list create the 1?
The terms that could add to 1 are
\begin{itemize}
\item $ \pm\frac{49}{4},\pm\frac{1}{4}, \pm\frac{1}{4}, \pm\frac{1}{4}  $  
\item or $ \pm\frac{7}{4},\pm\frac{7}{4}, \pm\frac{1}{4}, \pm\frac{1}{4}  $. 
\end{itemize}
Neither of these possibilities work.


\item $3,2,0,0$\\
How could this list create the 10?
The terms that could add to 10 are
\begin{itemize}
\item$ \pm9,\pm4, 0, 0  $,
\item$ \pm6,\pm6, 0, 0  $,
\item$ \pm6,0, 0, 0  $,
\item$ \pm4,0, 0, 0  $,
\item$ 0,0, 0, 0  $,
\item or $ \pm9,0, 0, 0  $.
\end{itemize}
None of these possibilities work.


\item $ \frac{5}{2},\frac{3}{2}, \frac{3}{2}, \frac{3}{2}  $\\
How could this list create the 1?
The terms that could add to 1 are
\begin{itemize}
\item$ \pm\frac{25}{4},\pm\frac{9}{4}, \pm\frac{9}{4}, \pm\frac{9}{4}  $  
\item or $ \pm\frac{15}{4},\pm\frac{15}{4}, \pm\frac{9}{4}, \pm\frac{9}{4}  $. 
\end{itemize}
Neither of these possibilities work.



\item $ \frac{5}{2},\frac{5}{2}, \frac{1}{2}, \frac{1}{2}  $\\
How could this list create the 1?
The terms that could add to 1 are
\begin{itemize}
\item$ \pm\frac{25}{4},\pm\frac{25}{4}, \pm\frac{1}{4}, \pm\frac{1}{4}  $,
\item$ \pm\frac{25}{4},\pm\frac{5}{4}, \pm\frac{5}{4}, \pm\frac{1}{4}  $,
\item or $ \pm\frac{5}{4},\pm\frac{5}{4}, \pm\frac{5}{4}, \pm\frac{5}{4}  $. 
\end{itemize}
None of these possibilities work.

\item $2,2,2,1$\\
How could this list create the 10?
The terms that could add to 10 are
\begin{itemize}
\item$ \pm4,\pm4, \pm4, \pm1  $  
\item or $ \pm4,\pm4, \pm2, \pm2  $.
\end{itemize}
Neither of these possibilities work.


\end {enumerate}


\section{An Elementary Way to Generate Pythagorean Quintuples}
\label{appendix_B}

This method was inspired by the method used in \cite{Halai}, which 
dealt with Pythagorean Quadruples. I strongly suspect that this concept
is a consequence of \cite{Roy_and_Sonia}.


Choose 3 nonnegative integers $a$, $b$, and $c$, excluding the case 
where two of them are odd and one is even.





Choose integers $p$ and $q$ that obey the following 
conditions:


\begin{itemize}
\item[(i)]
$p|(a^2 + b^2 + c^2)$ 


\item[(ii)]
$pq = a^2 + b^2 + c^2$ 



\item[(iii)]
$p \equiv q \pmod 2$



\end{itemize}




Let $d$ = $\frac{p-q}{2}$ and
$e$ = $\frac{p+q}{2}$. 
Note that if $p \not\equiv q \pmod 2$, then $d$ and $e$
would not be integers.

Then $a^2 + b^2 + c^2 + d^2 = e^2$. All Pythagorean quintuples can 
be generated in this way.


%\begin{comment}

\fix{I (as of June 27, 2019) believe the Heinz 57 result resulted from
an error in my computer code. I am also suspicious of the other table
in this section, and I don't see how it is directly relevant.}

\section{Testing the formulas from Theorem~\ref{thm:as_products}}
\label{appendix_C}






The author wrote a Python program to determine the number of
Pythagorean quintuples that Theorem~\ref{thm:as_products}
represents.  This program found all primitive Pythagorean quintuples
with all legs less than or equal to $20$.  The program then takes a
form, applies it to a set of four integers or half integers, and
records the resulting Pythagorean quintuple.


\begin{example}
Example of a generating list: 
\[ 4, 4, 0, 1
\] 
Example of a form:
\[
     (x+yi+zj+wk)(y+zi+wj-xk)
\]
The program applies the form: 
\[
     (4+4i+0j+1k)(4+0i+1j-4k)=20+15i+20j-8k
\]
The result:
\[
       8, 15, 20, 20
\]
The result is always a Pythagorean quintuple: 
\[8^2+ 15^2+ 20^2+ 20^2= 33^2\]
\end{example}





\begin{figure}
\[
\begin{array}{l|c||l|c}
1.\:\: (x+yi+zj+wk)(x+yi+zj+wk) &283&
30. \:\: (x+yi+zj+wk)(y+zi+xj-wk) &353\\
%
2. \:\: (x+yi+zj+wk)(x+yi+wj+zk) &349&
31. \:\: (x+yi+zj+wk)(y+zi+wj-xk) &405\\
%
3. \:\: (x+yi+zj+wk)(x+zi+yj+wk) &349&
32. \:\: (x+yi+zj+wk)(y+wi+xj-zk) &291\\
%
4. \:\: (x+yi+zj+wk)(x+zi+wj+yk) &361&
33. \:\: (x+yi+zj+wk)(z+xi+yj-wk) &353\\
%
5. \:\: (x+yi+zj+wk)(x+wi+yj+zk) &352&
34. \:\: (x+yi+zj+wk)(z+xi+wj-yk) &291\\[2mm]
%
6. \:\: (x+yi+zj+wk)(x+wi+zj+yk) &349&
35. \:\: (x+yi+zj+wk)(z+yi+xj-wk) &291\\
%
7. \:\: (x+yi+zj+wk)(y+xi+zj+wk) &379&
36. \:\: (x+yi+zj+wk)(z+wi+xj-yk) &5\\
%
8. \:\: (x+yi+zj+wk)(y+xi+wj+zk) &283&
37. \:\: (x+yi+zj+wk)(z+wi+yj-xk) &291\\
%
9. \:\: (x+yi+zj+wk)(y+zi+xj+wk) &352&
38. \:\: (x+yi+zj+wk)(w+xi+yj-zk) &405\\
%
10. \:\: (x+yi+zj+wk)(y+zi+wj+xk) &349&
39. \:\: (x+yi+zj+wk)(w+yi+zj-xk) &405\\[2mm]
%
11. \:\: (x+yi+zj+wk)(y+wi+xj+zk) &379&
40. \:\: (x+yi+zj+wk)(w+zi+xj-yk) &291\\
%
12. \:\: (x+yi+zj+wk)(y+wi+zj+xk) &393&
41. \:\: (x+yi+zj+wk)(w+zi+yj-xk) &321\\
%
13. \:\: (x+yi+zj+wk)(z+xi+yj+wk) &393&
42. \:\: (x+yi+zj-wk)(x+yi+zj+wk) &321\\
%
14. \:\: (x+yi+zj+wk)(z+xi+wj+yk) &349&
43. \:\: (x+yi+zj-wk)(x+zi+wj+yk) &353\\
%
15. \:\: (x+yi+zj+wk)(z+yi+xj+wk) &379&
44. \:\: (x+yi+zj-wk)(x+wi+yj+zk) &353\\[2mm]
%
16. \:\: (x+yi+zj+wk)(z+yi+wj+xk) &352&
45. \:\: (x+yi+zj-wk)(y+zi+xj+wk) &353\\
%
17. \:\: (x+yi+zj+wk)(z+wi+xj+yk) &283&
46. \:\: (x+yi+zj-wk)(z+xi+yj+wk) &353\\
%
18. \:\: (x+yi+zj+wk)(w+xi+zj+yk) &352&
47. \:\: (x+yi+zj-wk)(z+wi+xj+yk) &321\\
%
19. \:\: (x+yi+zj+wk)(w+yi+xj+zk) &393&
48. \:\: (x+yi+zj-wk)(w+zi+yj+xk) &321\\
%
20. \:\: (x+yi+zj+wk)(w+zi+xj+yk) &349&
49. \:\: (x+yi+zj-wk)(x+yi+wj-zk) &349\\ [2mm]
%
21. \:\: (x+yi+zj+wk)(w+zi+yj+xk) &283&
50. \:\: (x+yi+zj-wk)(x+zi+yj-wk) &349\\
%
22. \:\: (x+yi+zj+wk)(x+yi+zj-wk) &321&
51. \:\: (x+yi+zj-wk)(x+wi+zj-yk) &349\\
%
23. \:\: (x+yi+zj+wk)(x+yi+wj-zk) &405&
52. \:\: (x+yi+zj-wk)(y+xi+zj-wk) &379\\
%
24. \:\: (x+yi+zj+wk)(x+zi+yj-wk) &405&
53. \:\: (x+yi+zj-wk)(y+zi+wj-xk) &379\\
%
25. \:\: (x+yi+zj+wk)(x+zi+wj-yk) &353&
54. \:\: (x+yi+zj-wk)(y+wi+xj-zk) &349\\ [2mm]
%
26. \:\: (x+yi+zj+wk)(x+wi+yj-zk) &353&
55. \:\: (x+yi+zj-wk)(z+xi+wj-yk) &379\\
%
27. \:\: (x+yi+zj+wk)(x+wi+zj-yk) &291&
56. \:\: (x+yi+zj-wk)(z+wi+yj-xk) &349\\
%
28. \:\: (x+yi+zj+wk)(y+xi+zj-wk) &405&
57. \:\: (x+yi+zj-wk)(w+xi+yj-zk) &349\\
%
29. \:\: (x+yi+zj+wk)(y+xi+wj-zk) &321
\end{array}
\]
\caption{The $57$ possible forms of a Pythagorean quintuple in its primitive Hurwitz form as seen in Theorem~\ref{thm:as_products}. The number next to a form represents the number of times the program used the form to generate a small Pythagorean quintuple.} 
\label{fig:57}
\end{figure}



The author then
analyzed the forms to
see if they
capture all of the test set of primitive Pythagorean quintuples.
To select the forms in Figure~\ref{fig:57}
the author began with the quaternion
$(x + yi + zj + wk)^2$ with 
 $x$, $y$, $z$ and $w$ all positive.
Theorem~\ref{thm:as_products} required 
permutations and sign changes within the factors.
Without loss of generality, the first factor is
of the form $(x \pm yi \pm zj \pm wk)$,
while the second factor has all possible permutations
and sign changes of
 $x$, $y$, $z$ and $w$ to produce
$(x' + y'i + z'j + w'k)$.
This gives $2^7(4!)$ possible expressions.


The author let $x = \pi$, $y = e$,
$z = \sqrt{2}$ and $w = \sqrt{5}$ and evaluated
each of the $2^7(4!)$ expressions.
Of these, there were $57$ different resulting values.
The author chose one representative expression for each of the
$57$ values. Notice that there exists a great deal of 
redundancy among these $57$ expressions; any $2$ expressions
characterize roughly the same set of primitive
Pythagorean quintuples, at least for small examples. Intuitively,
this redundancy arises because each expression describes a way in which 
a quaternion can be factored. 
By Theorem~\ref{thm:unique_factorization}, a quaternion that 
represents a primitive Pythagorean quintuple can be factored in many 
ways, so many of these $57$ expressions characterize it.


\vanish{
%%% Old figure of Heinz 57
\begin{figure}
\begin{tabbing}
spacespacespacespacespacespacespacespacespace\=              \kill
1. $(x+yi+zj+wk)(x+yi+zj+wk)$ \>283\\
2. $(x+yi+zj+wk)(x+yi+wj+zk)$ \>349\\
3. $(x+yi+zj+wk)(x+zi+yj+wk)$ \>349\\
4. $(x+yi+zj+wk)(x+zi+wj+yk)$ \>361\\
5. $(x+yi+zj+wk)(x+wi+yj+zk)$ \>352\\
6. $(x+yi+zj+wk)(x+wi+zj+yk)$ \>349\\
7. $(x+yi+zj+wk)(y+xi+zj+wk)$ \>379\\
8. $(x+yi+zj+wk)(y+xi+wj+zk)$ \>283\\
9. $(x+yi+zj+wk)(y+zi+xj+wk)$ \>352\\
10. $(x+yi+zj+wk)(y+zi+wj+xk)$ \>349\\
11. $(x+yi+zj+wk)(y+wi+xj+zk)$ \>379\\
12. $(x+yi+zj+wk)(y+wi+zj+xk)$ \>393\\
13. $(x+yi+zj+wk)(z+xi+yj+wk)$ \>393\\
14. $(x+yi+zj+wk)(z+xi+wj+yk)$ \>349\\
15. $(x+yi+zj+wk)(z+yi+xj+wk)$ \>379\\
16. $(x+yi+zj+wk)(z+yi+wj+xk)$ \>352\\
17. $(x+yi+zj+wk)(z+wi+xj+yk)$ \>283\\
18. $(x+yi+zj+wk)(w+xi+zj+yk)$ \>352\\
19. $(x+yi+zj+wk)(w+yi+xj+zk)$ \>393\\
20. $(x+yi+zj+wk)(w+zi+xj+yk)$ \>349\\
21. $(x+yi+zj+wk)(w+zi+yj+xk)$ \>283\\
22. $(x+yi+zj+wk)(x+yi+zj-wk)$ \>321\\
23. $(x+yi+zj+wk)(x+yi+wj-zk)$ \>405\\
24. $(x+yi+zj+wk)(x+zi+yj-wk)$ \>405\\
25. $(x+yi+zj+wk)(x+zi+wj-yk)$ \>353\\
26. $(x+yi+zj+wk)(x+wi+yj-zk)$ \>353\\
27. $(x+yi+zj+wk)(x+wi+zj-yk)$ \>291\\
28. $(x+yi+zj+wk)(y+xi+zj-wk)$ \>405\\
29. $(x+yi+zj+wk)(y+xi+wj-zk)$ \>321\\
30. $(x+yi+zj+wk)(y+zi+xj-wk)$ \>353\\
31. $(x+yi+zj+wk)(y+zi+wj-xk)$ \>405\\
32. $(x+yi+zj+wk)(y+wi+xj-zk)$ \>291\\
33. $(x+yi+zj+wk)(z+xi+yj-wk)$ \>353\\
34. $(x+yi+zj+wk)(z+xi+wj-yk)$ \>291\\
35. $(x+yi+zj+wk)(z+yi+xj-wk)$ \>291\\
36. $(x+yi+zj+wk)(z+wi+xj-yk)$ \>5\\
37. $(x+yi+zj+wk)(z+wi+yj-xk)$ \>291\\
38. $(x+yi+zj+wk)(w+xi+yj-zk)$ \>405\\
39. $(x+yi+zj+wk)(w+yi+zj-xk)$ \>405\\
40. $(x+yi+zj+wk)(w+zi+xj-yk)$ \>291\\
41. $(x+yi+zj+wk)(w+zi+yj-xk)$ \>321\\
42. $(x+yi+zj-wk)(x+yi+zj+wk)$ \>321\\
43. $(x+yi+zj-wk)(x+zi+wj+yk)$ \>353\\
44. $(x+yi+zj-wk)(x+wi+yj+zk)$ \>353\\
45. $(x+yi+zj-wk)(y+zi+xj+wk)$ \>353\\
46. $(x+yi+zj-wk)(z+xi+yj+wk)$ \>353\\
47. $(x+yi+zj-wk)(z+wi+xj+yk)$ \>321\\
48. $(x+yi+zj-wk)(w+zi+yj+xk)$ \>321\\
49. $(x+yi+zj-wk)(x+yi+wj-zk)$ \>349\\
50. $(x+yi+zj-wk)(x+zi+yj-wk)$ \>349\\
51. $(x+yi+zj-wk)(x+wi+zj-yk)$ \>349\\
52. $(x+yi+zj-wk)(y+xi+zj-wk)$ \>379\\
53. $(x+yi+zj-wk)(y+zi+wj-xk)$ \>379\\
54. $(x+yi+zj-wk)(y+wi+xj-zk)$ \>349\\
55. $(x+yi+zj-wk)(z+xi+wj-yk)$ \>379\\
56. $(x+yi+zj-wk)(z+wi+yj-xk)$ \>349\\
57. $(x+yi+zj-wk)(w+xi+yj-zk)$ \>349
\end{tabbing}
\caption{The $57$ possible forms of a Pythagorean quintuple in its primitive Hurwitz form as seen in Theorem~\ref{thm:as_products}.} 
\label{fig:57}
\end{figure}
}



There are $337$ Pythagorean quintuples 
with all legs less than or equal to $20$.
See Figure~\ref{fig:quintuples}
for $50$ of those quintuples and
the number of ways to generate them:


\begin{figure}
\[
\begin{array}{l|c||l|c}
                  &\mbox{Times}    &   &\mbox{Times}\\ 
\mbox{Pythagorean Quintuple}     &\mbox{found}
&\mbox{Pythagorean Quintuple}     &\mbox{found}\\ \hline
&&&\\
0^2+ 0^2+ 0^2+ 1^2  = 1^2 & 255
& 2.5^2+ 3.5^2+ 6.5^2+ 19.5^2  = 21^2 &0\\
0^2+ 0^2+ 3^2+ 4^2  = 5^2 &180
&2.5^2+ 12.5^2+ 12.5^2+ 14.5^2  = 23^2 &36\\
%%
0^2+ 0^2+ 5^2+ 12^2  = 13^2 &180&
2.5^2+ 13.5^2+ 16.5^2+ 19.5^2  = 29^2 &0\\
%
0^2+ 0^2+ 8^2+ 15^2  = 17^2 &180&
3.5^2+ 5.5^2+ 8.5^2+ 10.5^2  = 15^2 &0\\
%
0^2+ 3^2+ 4^2+ 12^2  = 13^2 &252&
3.5^2+ 5.5^2+ 9.5^2+ 12.5^2  = 17^2 &0\\
%
1^2+ 2^2+ 8^2+ 10^2  = 13^2 &0&
3.5^2+ 5.5^2+ 15.5^2+ 18.5^2  = 25^2 &0\\
%
1^2+ 2^2+ 10^2+ 16^2  = 19^2 &0&
3.5^2+ 5.5^2+ 17.5^2+ 19.5^2  = 27^2 &0\\
%
1^2+ 4^2+ 4^2+ 4^2  = 7^2 &36&
3.5^2+ 6.5^2+ 6.5^2+ 8.5^2  = 13^2 &60\\
%
4^2+ 12^2+ 13^2+ 20^2  = 27^2 &0&
3.5^2+ 6.5^2+ 11.5^2+ 18.5^2  = 23^2 &0\\
%
4^2+ 13^2+ 16^2+ 20^2  = 29^2 &144&
3.5^2+ 7.5^2+ 10.5^2+ 10.5^2  = 17^2 &36\\
%
4^2+ 16^2+ 17^2+ 20^2  = 31^2 &0&
3.5^2+ 7.5^2+ 10.5^2+ 13.5^2  = 19^2 &0\\
%
8^2+ 13^2+ 14^2+ 14^2  = 25^2 &36&
3.5^2+ 7.5^2+ 11.5^2+ 15.5^2  = 21^2 &24\\
%
10^2+ 10^2+ 19^2+ 20^2  = 31^2 &36&
3.5^2+ 8.5^2+ 8.5^2+ 11.5^2  = 17^2 &60\\
%
12^2+ 16^2+ 17^2+ 20^2  = 33^2 &288&
4.5^2+ 4.5^2+ 7.5^2+ 8.5^2  = 13^2 &84\\
%
13^2+ 16^2+ 20^2+ 20^2  = 35^2 &36&
4.5^2+ 4.5^2+ 10.5^2+ 14.5^2  = 19^2 &84\\
%
13^2+ 20^2+ 20^2+ 20^2  = 37^2 &36&
4.5^2+ 4.5^2+ 13.5^2+ 17.5^2  = 23^2 &36\\
%
0.5^2+ 0.5^2+ 0.5^2+ 0.5^2  = 1^2 &30&
8.5^2+ 15.5^2+ 17.5^2+ 18.5^2  = 31^2 &0\\
%
0.5^2+ 0.5^2+ 1.5^2+ 2.5^2  = 3^2 &84&
9.5^2+ 10.5^2+ 19.5^2+ 19.5^2  = 31^2 &36\\
%
0.5^2+ 0.5^2+ 2.5^2+ 6.5^2  = 7^2 &84&
9.5^2+ 12.5^2+ 14.5^2+ 16.5^2  = 27^2 &48\\
%
0.5^2+ 0.5^2+ 3.5^2+ 3.5^2  = 5^2 &36&
10.5^2+ 10.5^2+ 11.5^2+ 16.5^2  = 25^2 &36\\
%
0.5^2+ 0.5^2+ 3.5^2+ 12.5^2  = 13^2 &84&
10.5^2+ 12.5^2+ 12.5^2+ 17.5^2  = 27^2 &36\\
%
0.5^2+ 0.5^2+ 5.5^2+ 9.5^2  = 11^2 &36&
10.5^2+ 16.5^2+ 16.5^2+ 17.5^2  = 31^2 &36\\
%
1.5^2+ 8.5^2+ 13.5^2+ 16.5^2  = 23^2 &0&
11.5^2+ 16.5^2+ 18.5^2+ 18.5^2  = 33^2 &36\\
%
1.5^2+ 10.5^2+ 10.5^2+ 17.5^2  = 23^2 &36&
12.5^2+ 12.5^2+ 17.5^2+ 18.5^2  = 31^2 &84\\
%
1.5^2+ 10.5^2+ 15.5^2+ 16.5^2  = 25^2 &0&
12.5^2+ 14.5^2+ 18.5^2+ 19.5^2  = 33^2 &48
\end{array}
\]
\caption{Generating program results for some small Pythagorean quintuples,}
\label{fig:quintuples}
\end{figure}


\vanish{
%\begin{figure}
\begin{tabbing}
spacespacespacespacespacespacespace\=    \kill
Pythagorean Quintuple     \>Number of ways to generate\\
$0^2+ 0^2+ 0^2+ 1^2  = 1^2$ \>$255$\\
$0^2+ 0^2+ 3^2+ 4^2  = 5^2$ \>$180$\\
$0^2+ 0^2+ 5^2+ 12^2  = 13^2$ \>$180$\\
$0^2+ 0^2+ 8^2+ 15^2  = 17^2$ \>$180$\\
$0^2+ 3^2+ 4^2+ 12^2  = 13^2$ \>$252$\\
$1^2+ 2^2+ 8^2+ 10^2  = 13^2$ \>$0$\\
$1^2+ 2^2+ 10^2+ 16^2  = 19^2$ \>$0$\\
$1^2+ 4^2+ 4^2+ 4^2  = 7^2$ \>$36$\\
$4^2+ 12^2+ 13^2+ 20^2  = 27^2$ \>$0$\\
$4^2+ 13^2+ 16^2+ 20^2  = 29^2$ \>$144$\\
$4^2+ 16^2+ 17^2+ 20^2  = 31^2$ \>$0$\\
$8^2+ 13^2+ 14^2+ 14^2  = 25^2$ \>$36$\\
$10^2+ 10^2+ 19^2+ 20^2  = 31^2$ \>$36$\\
$12^2+ 16^2+ 17^2+ 20^2  = 33^2$ \>$288$\\
$13^2+ 16^2+ 20^2+ 20^2  = 35^2$ \>$36$\\
$13^2+ 20^2+ 20^2+ 20^2  = 37^2$ \>$36$\\
$0.5^2+ 0.5^2+ 0.5^2+ 0.5^2  = 1^2$ \>$30$\\
$0.5^2+ 0.5^2+ 1.5^2+ 2.5^2  = 3^2$ \>$84$\\
$0.5^2+ 0.5^2+ 2.5^2+ 6.5^2  = 7^2$ \>$84$\\
$0.5^2+ 0.5^2+ 3.5^2+ 3.5^2  = 5^2$ \>$36$\\
$0.5^2+ 0.5^2+ 3.5^2+ 12.5^2  = 13^2$ \>$84$\\
$0.5^2+ 0.5^2+ 5.5^2+ 9.5^2  = 11^2$ \>$36$\\
$1.5^2+ 8.5^2+ 13.5^2+ 16.5^2  = 23^2$ \>$0$\\
$1.5^2+ 10.5^2+ 10.5^2+ 17.5^2  = 23^2$ \>$36$\\
$1.5^2+ 10.5^2+ 15.5^2+ 16.5^2  = 25^2$ \>$0$\\
$2.5^2+ 3.5^2+ 6.5^2+ 19.5^2  = 21^2$ \>$0$\\
$2.5^2+ 12.5^2+ 12.5^2+ 14.5^2  = 23^2$ \>$36$\\
$2.5^2+ 13.5^2+ 16.5^2+ 19.5^2  = 29^2$ \>$0$\\
$3.5^2+ 5.5^2+ 8.5^2+ 10.5^2  = 15^2$ \>$0$\\
$3.5^2+ 5.5^2+ 9.5^2+ 12.5^2  = 17^2$ \>$0$\\
$3.5^2+ 5.5^2+ 15.5^2+ 18.5^2  = 25^2$ \>$0$\\
$3.5^2+ 5.5^2+ 17.5^2+ 19.5^2  = 27^2$ \>$0$\\
$3.5^2+ 6.5^2+ 6.5^2+ 8.5^2  = 13^2$ \>$60$\\
$3.5^2+ 6.5^2+ 11.5^2+ 18.5^2  = 23^2$ \>$0$\\
$3.5^2+ 7.5^2+ 10.5^2+ 10.5^2  = 17^2$ \>$36$\\
$3.5^2+ 7.5^2+ 10.5^2+ 13.5^2  = 19^2$ \>$0$\\
$3.5^2+ 7.5^2+ 11.5^2+ 15.5^2  = 21^2$ \>$24$\\
$3.5^2+ 8.5^2+ 8.5^2+ 11.5^2  = 17^2$ \>$60$\\
$4.5^2+ 4.5^2+ 7.5^2+ 8.5^2  = 13^2$ \>$84$\\
$4.5^2+ 4.5^2+ 10.5^2+ 14.5^2  = 19^2$ \>$84$\\
$4.5^2+ 4.5^2+ 13.5^2+ 17.5^2  = 23^2$ \>$36$\\
$8.5^2+ 15.5^2+ 17.5^2+ 18.5^2  = 31^2$ \>$0$\\
$9.5^2+ 10.5^2+ 19.5^2+ 19.5^2  = 31^2$ \>$36$\\
$9.5^2+ 12.5^2+ 14.5^2+ 16.5^2  = 27^2$ \>$48$\\
$10.5^2+ 10.5^2+ 11.5^2+ 16.5^2  = 25^2$ \>$36$\\
$10.5^2+ 12.5^2+ 12.5^2+ 17.5^2  = 27^2$ \>$36$\\
$10.5^2+ 16.5^2+ 16.5^2+ 17.5^2  = 31^2$ \>$36$\\
$11.5^2+ 16.5^2+ 18.5^2+ 18.5^2  = 33^2$ \>$36$\\
$12.5^2+ 12.5^2+ 17.5^2+ 18.5^2  = 31^2$ \>$84$\\
$12.5^2+ 14.5^2+ 18.5^2+ 19.5^2  = 33^2$ \>$48$\\
\end{tabbing}
%\label{example_quintuples}
%\end{figure}
}

%\end{comment}


\end{appendices}





%\newpage

\newcommand{\journal}[6]{{\sc #1,} #2, {\it #3} {\bf #4} (#5), #6.}
\newcommand{\preprint}[3]{{\sc #1,} #2, preprint #3.}
\newcommand{\book}[4]{{\sc #1,} #2, #3, #4.}
\newcommand{\collection}[6]{{\sc #1,}  #2, #3, in {\it #4}, #5, #6.}
\newcommand{\JCTA}{J.\ Combin.\ Theory Ser.\ A}
\newcommand{\arxiv}[3]{{\sc #1,} #2, {\tt #3}.}
\newcommand{\article}[3]{{\sc #1,} #2, {\tt #3}.}






\begin{thebibliography}{1}



\bibitem{Baez}
\journal{J.\ C.\ Baez}
        {Review of ``On Quaternions and Octonions: 
         Their Geometry, Arithmetic and Symmetry''}
        {Bull.\ Amer.\ Math.\ Soc.}{42}{2005}{229--243}
        


\bibitem{Barning}
\journal{F.\  J.\  M.\  Barning} 
      {On Pythagorean and quasi-Pythagorean triangles and a
      generation process with the help of unimodular matrices (Dutch)}
      {Math.\ Centrum Amsterdam Afd.\ Zuivere Wisk.}
      {ZW-011}{1963}{37 pp}



\bibitem{Berggren}
\journal{B.\ Berggren} 
        {Pytagoreiska trianglar (Swedish)} 
        {Elementa: Tidskrift f\"or element\"ar
        matematik, fysik och kemi.}
        {17}{1934}{129--139}


\bibitem{Conrad}
\journal{K.\ Conrad}
        {Pythagorean descent}
        {}
        {}{Preprint}{1-9}



\bibitem{Conway_and_Smith}
\book{J.\ Conway and D.\ Smith}
     {On Quaternions and Octonions: 
      Their Geometry, Arithmetic and Symmetry}
     {A K Peters, Ltd., Natick MA}
     {2003}


\bibitem{Davidoff_Sarnak_Valette}
\book{G.\ Davidoff, P.\ Sarnak, and A.\ Valette}
         {Elementary Number Theory,
           Group Theory,
           and Ramanujan Graphs}
         {Cambridge University Press}
         {2003}

\begin{comment}


\bibitem{Dickson}
\book{L.\ E.\ Dickson}
         {History of the Theory of Numbers}
         {AMS Chelsea Publishing}
         {1999}

\bibitem{Ehrenborg_2018}
\preprint{T.\ Ehrenborg}
        {Pythagorean Quintuples and Quaternions}
        {2018}




\bibitem{Ferrari}
\journal{F.\ Ferrari}
        {Risoluzione Dell'Equazione}
        {Supplemento al Periodico di Matematica}
        {11}{1908}{129--131}

\end{comment}

\bibitem{Frisch_and_Vaserstein}
\journal{S.\ Frisch and L.\ Vaserstein}
       {Polynomial parametrization of Pythagorean quadruples,
        quintuples and sextuples}
       {J.\ Pure Appl.\ Algebra}{216}{2012}{184--191}


\bibitem{Guzeltepe}
\journal{M.\ G\"uzeltepe}
        {Codes over Hurwitz integers}
        {Discrete Math.}
        {313}{2013}{704--714}



\bibitem{Halai}
\article{C.\ Halai}
      {Triples and Quadruples: From Pythagoras to Fermat}
      {plus.maths.org/ content/triples-and-quadruples}{}

\bibitem{Hamilton}
\journal{W.\ Hamilton}
        {On Quaternions; or on a new System of Imaginaries in Algebra}
        {The London, Edinburgh and Dublin Philosophical Magazine
         and Journal of Science}
        {25}{1844}{10--13}


\bibitem{Hardy_and_Wright}
\book{G.\ H.\ Hardy and E.\ M.\ Wright}
         {An Introduction to the Theory of Numbers, 6th Edition}
         {Oxford University Press, Oxford} 
         {2008}




\bibitem{Hatcher}
\book{A.\ Hatcher}
         {Topology of Numbers}
         {Manuscript}
         {February 2018 version}



\bibitem{Heath}
\book{T.\ L.\ Heath}
     {Diophantus of Alexandria, 2nd Edition}
     {Cambridge University Press, London}
     {1910}

\begin{comment}

\bibitem{Hurwitz}
\journal{A.\ Hurwitz}
        {\"Uber die Komposition der quadratischen Formen}
        {Mathematische Annalen}
        {88}{1922}{1--25}

\end{comment}


\bibitem{Jacobi}
\journal{C.\ G.\ J.\ Jacobi}
        {De compositione numerorum e quatuor quadratis}
        {Journal f\"ur die reine und angewandte Mathematik}
        {12}{1834}{167--172}

\bibitem{Euclid}
\article{D.\ E.\ Joyce}
        {Euclid's Elements, Book X, Proposition 29}
        {\\https://mathcs.clarku.edu/\~{}djoyce/java/elements/elements.html}



\bibitem{Lalin}
\article{M.\ N.\ Lal\'in}
      {Every Positive Integer is the Sum of Four Squares! (and other
      exciting problems)} 
      {Sophex –-- University of Texas at Austin, 6
      pp}{}

\begin{comment}

\bibitem{Mordell}
\book{L.\ J.\ Mordell}
         {Diophantine Equations}
         {Academic Press, London and New York} 
         {1969}

\end{comment}


\bibitem{Roy_and_Sonia}
\arxiv{T.\ Roy and F.\ J.\ Sonia}
      {A direct method to generate Pythagorean triples and its
       generalization to Pythagorean quadruples and $n$-tuples}
      {arXiv:1201.2145 [math.NT], 11 pp}



\end{thebibliography}





\end{document}










