%%%%%%%%%%%%%%%%%%%%%%%%%%%%%%%%%%
%
%
%  Classifying Quaternion Identities
%  (journal article)
%  
%  by Theodore Ehrenborg
%
%
% 
% Last edited: April 29, 2019
%
%
%%%%%%%%%%%%%%%%%%%%%%%%%%%%%%%%%
%
%%%%%%%%%%%%%%%%%%%%%%%%%%%%%%%%%



%%%%%%%%%%%%%%%%%%%%%%%%%%%%%%%%%
%
% pdf settings
%
%%%%%%%%%%%%%%%%%%%%%%%%%%%%%%%%%
%
\pdfpagewidth=8.5truein
\pdfpageheight=11truein


%
%%%%%%%%%%%%%%%%%%%%%%%%%%%%%%%%%



\documentclass[12pt]{article}
%\usepackage{anyfontsize}
\usepackage{amssymb, amsmath, fullpage, amsthm}
\usepackage{mathrsfs}
\usepackage{tikz}
\usepackage[title]{appendix}
\usepackage{todonotes}
\usepackage{mathrsfs}
\usepackage{ gensymb }
\usepackage{enumerate}
\usepackage{caption}


\usetikzlibrary{math}




\parskip2mm

\newtheorem{theorem}{Theorem}[section]
\newtheorem{hypothesis}[theorem]{Hypothesis}
\newtheorem{lemma}[theorem]{Lemma}
\newtheorem{proposition}[theorem]{Proposition}
\newtheorem{corollary}[theorem]{Corollary}
\newtheorem{remarks}[theorem]{Remarks}


\theoremstyle{definition}
\newtheorem{definition}[theorem]{Definition}
\newtheorem{example}[theorem]{Example}



\hyphenation{Hurwitz}

\font\german = eufm10 scaled\magstep1
\font\Cp = msbm10

\newcommand{\Ccc}{\hbox{\Cp C}}
\newcommand{\Fff}{\hbox{\Cp F}}
\newcommand{\Hhh}{\hbox{\Cp H}}
\newcommand{\Nnn}{\hbox{\Cp N}}
\newcommand{\Rrr}{\hbox{\Cp R}}
\newcommand{\Sss}{\hbox{\Cp S}}
\newcommand{\Zzz}{\hbox{\Cp Z}}


\newcommand{\vanish}[1]{}
\newcommand{\coveredby}{\prec}
\newcommand{\divides}{\mid}
\newcommand{\notdivides}{\nmid}
\newcommand{\timesdots}{\times \cdots \times}

\numberwithin{equation}{section}
%\usepackage[bindingoffset=0.2in,
%            left=1in,right=1in,top=1in,bottom=1in,
%            footskip=.25in]{geometry}%Sets margins
%\pagenumbering{gobble}%No page numbers




 


\DeclareMathOperator{\inv}{inv}
\DeclareMathOperator{\frst}{frst}
\DeclareMathOperator{\er}{er}
\DeclareMathOperator{\asc}{asc}
\DeclareMathOperator{\odd}{odd}

\newcommand{\ascodd}{\asc_{\odd}}

\newcommand{\doubleprime}{\prime\prime}




\newcommand{\SSSS}{\mathfrak{S}}

\newcommand{\Gaussian}[2]{\genfrac{[}{]}{0pt}{}{#1}{#2}_q}

\begin{document}
%\begin{landscape}


\title{Classifying Quaternion Identities}


\author{\sc Theodore EHRENBORG%\thanks{Corresponding author:
%Department of Mathematics,
%University of Kentucky,
%Lexington, KY 40506-0027,
%USA,
%{\tt theodore.ehrenborg@gmail.com}.}
%\:\: 
%\:\:
}



\date{}
%%%\date{Last edited on \today}

\maketitle



\begin{abstract}
This number theory project investigates identities found by
multiplying together quaternions in \( \mathbb{L}[x,y,z,w] \), the
Lipschitz quaternions \( \mathbb{L} \) adjoined with the
indeterminates \(x\), \(y\), \(z\), \(w\).  Recall that quaternions
are \(4\)-dimensional complex numbers.  These identities provide
solutions to \( \sum_{j = 1}^{p} \tau_j ^ 2 = \left( \sum_{i = 1}^{m}
x_i ^ 2 \right) ^ n \). We present a rigorous definition that captures
the intuitive notion of when two such identities are equivalent. This
definition implies that the true structure of this problem involves the
group action of the direct product \( \mathfrak{S}_4^\pm \times \mathfrak{S}_4^\pm \).  Using
two complementary methods, we compute the number of equivalence
classes for \(n = 1, 2, 3, 4,\) where \(n\) is the number of
quaternion factors. We move to the case concerning products of complex
numbers, namely \( \mathbb{Z}[i][x,y] \). Using the fact that the
Gaussian integers are commutative under multiplication, we
characterize these equivalence classes, thus also providing an
enumeration.


\end{abstract}





\section{Introduction}

\todo[inline]{Define quaternions, Lipschitz quaternions,
norm of a quaternion and complex numbers, complex numbers}


The following groups are an essential part of Definition~\ref{def:general}
and Definition~\ref{def:2D}.
\begin{definition}
The {\em symmetric group} \( \mathfrak{S}_n \) is the 
set of all permutations \( \pi = \pi_1 \cdots \pi_n \) 
of the \( n \) element set \( \{ 1, 2, \ldots, n \} \),
where \( \pi(i) = \pi_i \).
The {\em signed symmetric group} \( \mathfrak{S}_n^\pm \)
is the set of all permutations \( \sigma = \sigma_1 \cdots \sigma_n\)
of the set \( \{ \pm 1, \pm 2, \ldots, \pm n \} \) such that
\( | \sigma | = | \sigma_1 | \cdots |\sigma_n| \in \mathfrak{S}_n \).
\end{definition}

\todo[inline]{How to compose two elements in signed symmetric group?}

For a signed permutation \( \pi \in \mathfrak{S}_m^\pm \), let \( \pi \) act on a polynomial in the 
\(m\) variables \( x_1,x_2, \ldots, x_m \) by sending \( x_j \) to 
\[
\pi(x_j) =
\begin{cases}
x_{\pi_j} & \text{if } \pi_j > 0 \\
-x_{-\pi_j} & \text{if } \pi_j < 0
\end{cases}
\]

\begin{definition}
\label{def:general}
Fix \( p, m \in \mathbb{N} \). 
Let \( \tau = ( \tau_1, \ldots, \tau_p) \)
be a tuple of length \( p \) where 
\( \tau_i \in \mathbb{Z}[x_1,x_2, \ldots, x_m] \), where \( i = 1, \ldots, p \).
We define an equivalence relation, denoted by \( \simeq \), on \(p\)-tuples
\( \tau \) 
by taking the transitive closure of the following three relations:
\begin{itemize}
\item
\( ( \tau_1, \ldots, \tau_p) \simeq ( \tau'_1, \ldots, \tau'_p) \)
if there exists a signed permutation \( \pi \in \mathfrak{S}_m^\pm \)
acting on the \( m \) variables such that \( \pi( \tau_i ) = \tau'_i \),
 where \( i = 1, \ldots, p \).
\item
\( ( \tau_1, \ldots, \tau_p) \simeq ( \tau'_1, \ldots, \tau'_p) \)
if there exists a permutation \( \sigma \in \mathfrak{S}_p \)
such that \( \tau_{\sigma(i)} = \tau'_i \), where \( i = 1, \ldots, p \).
\item
\( ( \tau_1, \ldots, \tau_p) \simeq ( \pm \tau_1, \ldots, \pm \tau_p) \)
\end{itemize}
\end{definition}


\begin{example}
Definition~\ref{def:general} implies that the following is true:
\begin{align*}
( xz,\: y^2,\: yz )  
&\simeq ( (-y)(-x),\: z^2,\: z(-x) ) \\
&\simeq ( (-y)(-x),\: z(-x),\: z^2 ) \\
&\simeq ( -(-y)(-x),\: z(-x),\: -z^2 ) 
\end{align*}
\end{example}



%Fix \( p \in \mathbb{Z} \). 
We are interested in counting the number of
equivalence classes of the set of all tuples \( \tau \) where 
%the sum of the squares of the elements of \( \tau \) is
\[
\sum_{j = 1}^{p}  \tau_j ^ 2  
= 
\left( \sum_{i = 1}^{m}  x_i ^ 2  \right) ^ n 
\] 
This problem is most easily attacked when we
view \( \left( \sum_{i = 1}^{m}  x_i ^ 2  \right) ^ n \)
as the norm of a product of complex numbers or quaternions.
Thus we will focus on the cases where \( p = m = 2 \) and where \( p = m = 4 \).
The general problem can also be viewed as finding the disjoint orbits of 
various tuples, where the group action is \( \mathfrak{S}_p^\pm \times \mathfrak{S}_m^\pm \).



\begin{example}
\bf
Consider the case where \( p = m = 2 \) and \( n = 2\).
The following two identities are representatives from 
the two different equivalence classes in this case. These identities
were generated by a product of complex numbers. 

\noindent
Identity i:
\begin{equation*}
(x + iy)(x + iy) = (x^2 - y^2 ) + i(2xy) 
\end{equation*}
Taking norms, this gives the identity:
\begin{equation*}
    (x^2 - y^2 )^2 + (2xy)^2 
    = (x^2 + y^2)^2
\end{equation*}
Identity ii:
\begin{equation*}
    (x + iy )(x - iy )
    = (x^2 + y^2 ) + i(0)  
\end{equation*}
Taking norms, this gives the identity:
\begin{equation*}
    (x^2 + y^2 )^2 + (0)^2
    = (x^2 + y^2 )^2
\end{equation*}
\end{example}



\begin{example}
\bf
Consider the case where \( p = m = 2 \) and \( n = 3\).
The following two identities are representatives from 
the two different equivalence classes in this case. These identities
were generated by a product of complex numbers. 

\noindent
Identity iii:
\begin{align*}
    (x + iy)(x + iy)(x + iy) 
    = x(x^2 - 3y^2) + i(  y(3x^2 - y^2) )  
    \end{align*}
Taking norms, this gives the identity:
    \begin{align*}
    (x(x^2 - 3y^2))^2 + (  y(3x^2 - y^2) )^2  
    = (x^2 + y^2)^3
    \end{align*}
Identity iv:
    \begin{align*}
    (x + iy )(x + iy)(x - iy ) 
    = x(x^2 + y^2 ) + i(y(x^2 + y^2))  
    \end{align*}
Taking norms, this gives the identity:
    \begin{align*}
    ( x(x^2 + y^2) )^2 + ( y(x^2 + y^2) )^2 
    = (x^2 + y^2 )^3
    \end{align*}
\end{example}







\section{The case where \( p = m = 2\)}


In the case where \( p = m = 2\), Definition~\ref{def:general} has an alternate form.



\begin{definition}
\label{def:2D}

Let \( h(z), h'(z) \in \mathbb{Z} [i][x,y] \),
where \( z = x + iy \).
Let \( M \) be the following set of mappings: 
\[
M = \{ z \mapsto uz \mid u \in \{ \pm 1, \pm i \} \}  
\cup \{ z\mapsto u \bar{z} \mid u \in \{ \pm 1, \pm i \} \}  
\]
%Write \( f(x,y) + i \cdot g(x,y) \) and \( f'(x,y) + i \cdot g'(x,y) \) as  \( h(z) \) and  \( h'(z) \), respectively.
We say \( h(z) \simeq h'(z) \) when there exist mappings \( \varphi, \varphi' \in M \)
such that \( \varphi( h( \varphi'( z ) ) )  = h'(z) \).

%\todo{ Prove that this is an equivalence relation. 
%That is, it has the reflexive, symmetric, and transitive properties. }

\end{definition}

\begin{lemma}
\bf
The relation \( \simeq \) is an equivalence relation.
\end{lemma}

\begin{proof}
\bf
The relation \( \simeq \) satisfies the three conditions of an equivalence relation.
\begin{enumerate}
\item Reflexive Property: If we choose \( \mu \) and \( \mu'\) to be the identity map \( z \mapsto z \), 
then \( h(z) \simeq h(z) \).

\item Symmetric Property: Let  \( h(z) \simeq h'(z) \), that is, there exist mappings 
\( \mu, \mu' \in M \)
such that \( \mu( h( \mu'( z ) ) )  = h'(z) \).
\(M\) is isomorphic to the signed symmetric group \( \mathfrak{S}_2^\pm \), so every mapping in \(M\)
has an inverse in \(M\). As  \( \mu^{-1}( h'( (\mu')^{-1}( z ) ) )  = h(z) \),
we have \( h'(z) \simeq h(z) \).

\item Transitive Property: Let \( a(z) \simeq b(z) \) and \( b(z) \simeq c(z) \). By the 
Symmetric Property, we have \( c(z) \simeq b(z) \). Thus there exist mappings
\( \mu, \mu', \nu, \nu' \in M \) such that \( \mu( a( \mu'( z ) ) )  = b(z) \) and 
\( \nu( c( \nu'( z ) ) )  = b(z) \). As a result:
\[
\mu( a( \mu'( z ) ) ) = \nu( c( \nu'( z ) ) )  
\]
Thus we have: 
\[
a(  z  ) = \mu^{-1}( \nu( c( (\mu')^{-1}( \nu'( z ) ) ) ) )
\]
This means that \( c(z) \simeq a(z) \) or \( a(z) \simeq c(z) \).
\end{enumerate}
\end{proof}
Recall that given \( f(x,y), g(x,y) \in \mathbb{Z}[x,y] \),
we can find \( h(z) \in \mathbb{Z} [i][x,y] \)
such that
\( z = x + iy \)
and
\( h(z) = f(x,y) + i \cdot g(x,y) \). 
The converse is also true.



\begin{lemma}
\bf Let \( h(z) \simeq h'(z) \), where \( h(z), h'(z) \in \mathbb{Z}[i][x,y] \)
 and \( z = x+ iy \). Suppose \( ( x^2 + y^2 ) ^ u
    \divides h(z) \), where \( u \in \mathbb{N} \). Then \( ( x^2 + y^2 ) ^ u \divides h'(z) \).
\end{lemma}
   
\begin{proof}
\bf
Let \( h(z) = ( x^2 + y^2 ) ^ u  p(z) \), where \( p(z) \in \mathbb{Z} [i][x,y] \). 
Whatever mappings we apply to \(h\) and \(z\)
to get \( h'(z) \), we also apply to the factors of \( h(z) \). 
No mapping will remove the factor of \( ( x^2 + y^2 ) ^ u \).

\end{proof}

\begin{corollary}
\bf
Let \( h(z), h'(z) \in \mathbb{Z}[i][x,y] \) with \( z = x+ iy \).
\[
h(z) \simeq h'(z) 
\implies 
( \forall u \in \mathbb{N} \cup \{ 0 \},  ( x^2 + y^2 ) ^ u \divides h(z)  \iff ( x^2 + y^2 ) ^ u \divides h'(z)  )
\]
\end{corollary}




\begin{theorem}
\bf
Consider the set of all \(2\)-tuples \( \tau \) where 
\(
  \tau_1 ^ 2   +   \tau_2 ^ 2   
= 
\left(  x_1 ^ 2 + x_2 ^ 2  \right) ^ n 
\).
The number of equivalence classes within this set 
is exactly  \( \lfloor \frac{n}{2} \rfloor + 1 \). 
Moreover, each equivalence class contains a tuple
of the form \( ( \Re( \beta ) , \Im( \beta ) ) \),
where \( \beta = (x + iy)^j (x -  iy)^{n-j} \),
with \( j \)
being one of \( 0, 1, 2, \ldots, \lfloor \frac{n}{2} \rfloor \).
\end{theorem}



\begin{proof}
\bf
Suppose 
\[ (x^2 + y^2)^n = f(x,y)^2 + g(x,y)^2 \]
where \( f(x,y), g(x,y) \in \mathbb{Z}[x,y] \).
Then  
\[ (x + iy)^n (x -  iy)^n = ( f(x,y) + i \cdot g(x,y) ) ( f(x,y) - i \cdot g(x,y) ) .\]
Since \( \mathbb{Z} [i][x,y] \) is a unique factorization domain, 
\[ f(x,y) + i \cdot g(x,y) = (x + iy)^j (x -  iy)^k ( \pm 1 \text{ or} \pm i ) \]
and
\[ f(x,y) - i \cdot g(x,y) = (x + iy)^r (x -  iy)^s ( \pm 1 \text{ or} \mp i ), \]
where \( j, k, r, s \in \mathbb{N} \cup \{0\} \). 
We know \( j + r = n \) and \( k + s = n \), as well as (by taking norms) \( j + k = n \) and \( r + s = n \).

Thus \(r = k\) and \(s = j\).

Clearly \( f(x,y) + i \cdot g(x,y) \simeq  f(x,y) - i \cdot g(x,y) \).
Since 
\[ 
f(x,y) + i \cdot g(x,y) = (x + iy)^j (x -  iy)^{n-j} ( \pm 1 \text{ or} \pm i ) 
\]
and 
\[ 
f(x,y) - i \cdot g(x,y) = (x + iy)^{n-j} (x -  iy)^j ( \pm 1 \text{ or} \mp i ) ,
\]
each equivalence class contains a representative with \( j \leq n - j
\).  As \( 2j \leq n \), we have \( j \leq \frac{n}{2} \), so \( j \)
is one of \( 0, 1, 2, \ldots, \lfloor \frac{n}{2} \rfloor \). This shows
that there are at most \( \lfloor \frac{n}{2} \rfloor + 1 \)
equivalence classes. Now we show that there are at least that many.



Consider \( (x + iy)^v (x -  iy)^{n-v} \)  and \( (x + iy)^u (x -  iy)^{n-u} \),
where \( v \neq u \) and \(v,u \leq [ \frac{n}{2} ] \). We have:
\begin{align*}
(x + iy)^v (x -  iy)^{n-v} &= (x^2 + y^2)^v (x -  iy)^{n-2v}
\\
(x + iy)^u (x -  iy)^{n-u} &= (x^2 + y^2)^u (x -  iy)^{n-2u}
\end{align*}

Without loss of generality, \( v > u \). 
Since \( \mathbb{Z} [i][x,y] \) is a unique factorization domain,
\( (x^2 + y^2) \notdivides (x -  iy)^m \) for \( m \in \mathbb{N} \cup \{ 0 \} \).
Thus \( (x^2 + y^2)^v \divides (x + iy)^v (x -  iy)^{n-v} \)
but \( (x^2 + y^2)^v \notdivides (x + iy)^u (x -  iy)^{n-u} \).

Therefore \( (x + iy)^v (x -  iy)^{n-v} \not\simeq (x + iy)^u (x -  iy)^{n-u} \), 
so the \( \lfloor \frac{n}{2} \rfloor + 1 \) representatives come from
distinct equivalence classes.

Thus there are \( \lfloor \frac{n}{2} \rfloor + 1 \) equivalence classes, 
one each for \( v = 0, 1, 2, \ldots,  \lfloor \frac{n}{2} \rfloor \).
\end{proof}









\begin{figure}[h]
\label{fig:2D}

\begin{center}
\begin{tikzpicture}

\tikzmath{
\hash = 0.1;
\height = 6;
\length = 10;
\scale = 2;
}


\draw[thick] (0,0) -- (\scale * \length,0);
\fill (\scale * \length/2, -1) circle (0cm) node[anchor=north] {\Large \( n \) };

\draw[thick] (0,0) -- (0,\scale * \height);
%\fill (-2, \scale * \height/2 ) circle (0cm) node[rotate=90,anchor=south] {\Large  Number of };
%\fill (-1, \scale * \height/2 ) circle (0cm) node[rotate=90,anchor=south] {\Large  equivalence classes };
\fill (-1, \scale * \height/2 ) circle (0cm) node[anchor=east] {\Large  \( \kappa \) };


\foreach \y in {1, ..., \height}
    \draw[thick] ( -\hash, \scale * \y ) node[anchor=east] {\Large \y} -- ( \hash, \scale * \y );

\foreach \x in {1, ..., \length}
{
   \draw[thick] ( \scale * \x, -\hash ) node[anchor=north] {\Large \x} -- ( \scale * \x, \hash ) ;
   \fill (\scale * \x, { \scale *  floor( \x / 2 ) + \scale } ) circle (.1cm);
}
  

\end{tikzpicture}

\end{center}

\caption{
\bf The number of equivalence classes \( \kappa \)
when \(p = m = 2\) is \( \lfloor \frac{n}{2} \rfloor + 1 \) .
 }
\end{figure}





\begin{table}[h]

\label{table:4D}


\begin{center}


\begin{tabular}{ c | c }
 \( n \) & \( \kappa \) \\
\hline\hline
 1 & 1 \\
\hline
 2 & 8 \\
\hline
 3 & 48 \\
\hline
 4 & 965 
\end{tabular}





\end{center}
\caption{
The conjectured number of equivalence
classes \( \kappa \)  
in the case where \(p = m = 4\).
 }

\end{table}



\section{The case where \(p = m = 4\) and \(n = 2\)}








%Consider the following eight expressions, each of which is in
%\( \mathbb{Z}[i,j,k][x,y,z,w] \).
The following eight identities are representatives from 
eight different equivalence classes. These identities
were generated by a product of quaternions. 

\begin{enumerate}[{Identity} I:]
\item
    \begin{align*}
    &(x + iy + jz + kw)(x + iy + jz + kw) \\
    &= (x^2 - y^2 - z^2 - w^2 ) + i(2xy) + j(2xz) + k(2xw) 
    \end{align*}
Taking norms, this gives the identity:
    \begin{align*}
    &(x^2 - y^2 - z^2 - w^2 )^2 + (2xy)^2 + (2xz)^2 + (2xw)^2 \\
    &= (x^2 + y^2 + z^2 + w^2)^2
    \end{align*}
\item
    \begin{align*}
    &(x + iy + jz + kw)(x + iy + jz - kw) \\
    &= (x^2 - y^2 - z^2 + w^2 ) + i(2xy - 2zw) + j(2xz + 2yw) + k(0) 
    \end{align*}
Taking norms, this gives the identity:
    \begin{align*}
    &(x^2 - y^2 - z^2 + w^2 )^2 + (2xy - 2zw)^2 + (2xz + 2yw)^2 + (0)^2\\ 
    &= (x^2 + y^2 + z^2 + w^2)^2
    \end{align*}
\item
    \begin{align*}
    &(x + iy + jz + kw)(x - iy - jz - kw) \\
    &= (x^2 + y^2 + z^2 + w^2 ) + i(0) + j(0) + k(0) 
    \end{align*}
Taking norms, this gives the identity:
    \begin{align*}
    &(x^2 + y^2 + z^2 + w^2 )^2 + (0)^2 + (0)^2 + (0)^2 \\
    &= (x^2 + y^2 + z^2 + w^2)^2
    \end{align*}
\item
    \begin{align*}
    &(x + iy + jz + kw)(x + iy + jw + kz) \\
    &= (x^2 - y^2 - 2zw ) + i(2xy + z^2 -w^2) \\
        &+ j(xz - yz + xw + yw) + k(xz - yz + xw + yw) 
    \end{align*}
Taking norms, this gives the identity:
    \begin{align*}
    &(x^2 - y^2 - 2zw )^2 + (2xy + z^2 - w^2)^2 \\
        &+ (xz - yz + xw + yw)^2 + (xz - yz + xw + yw)^2 \\
    &= (x^2 + y^2 + z^2 + w^2)^2
    \end{align*}
\item
    \begin{align*}
    &(x + iy + jz + kw)(x + iy + jw - kz) \\
    &= (x^2 - y^2 ) + i(2xy - z^2 - w^2) \\
        &+ j(xz + yz + xw + yw) + k(-xz - yz + xw + yw) 
    \end{align*}
Taking norms, this gives the identity:
    \begin{align*}
    &(x^2 - y^2 )^2 + (2xy - z^2 - w^2)^2 \\
        &+ (xz + yz + xw + yw)^2 + (-xz - yz + xw + yw)^2 \\ 
    &= (x^2 + y^2 + z^2 + w^2)^2
    \end{align*}
\item
    \begin{align*}
    &(x + iy + jz + kw)(x - iy + jw - kz) \\
    &= (x^2 + y^2 ) + i(- z^2 - w^2) \\
        &+ j(xz + yz + xw - yw) + k(-xz + yz + xw + yw) 
    \end{align*}
Taking norms, this gives the identity:
    \begin{align*}
    &(x^2 + y^2 )^2 + (- z^2 - w^2)^2 \\
        &+ (xz + yz + xw - yw)^2 + (-xz + yz + xw + yw)^2 \\
    &= (x^2 + y^2 + z^2 + w^2)^2
    \end{align*}
\item
    \begin{align*}
    &(x + iy + jz + kw)(x + iz + jw + ky) \\
    &= (x^2 - yz - yw - zw ) + i(xy + xz + yz - w^2) \\
        &+ j(-y^2 + xz + xw + zw) + k(xy - z^2 + xw + yw) 
    \end{align*}
Taking norms, this gives the identity:
    \begin{align*}
    &(x^2 - yz - yw - zw )^2 + (xy + xz + yz - w^2)^2 \\
        &+ (-y^2 + xz + xw + zw)^2 + (xy - z^2 + xw + yw)^2 \\
    &= (x^2 + y^2 + z^2 + w^2)^2
    \end{align*}
\item
    \begin{align*}
    &(x + iy + jz + kw)(x + iz + jw - ky) \\
    &= (x^2 - yz + yw - zw ) + i(xy + xz - yz - w^2) \\
        &+ j(y^2 + xz + xw + zw) + k(xw - xy - z^2  + yw) 
    \end{align*}
Taking norms, this gives the identity:
    \begin{align*}
    &(x^2 - yz + yw - zw )^2 + (xy + xz - yz - w^2)^2 \\
        &+ (y^2 + xz + xw + zw)^2 + (xw - xy - z^2  + yw)^2 \\
    &= (x^2 + y^2 + z^2 + w^2)^2
    \end{align*}
\end{enumerate}











\section{Conclusion}



The original
goal of this project was to find a parameterization, preferably a polynomial one,  of 
Pythagorean~quintuples in four parameters.
Such a parameterization of quintuples
has been done using $12$ parameters~\cite{Polynomial_parametrization},
but the method involved a
specific case of sextuples, not a direct relationship between 
quintuples and quaternions. 
This project has shown that a
direct relationship is not trivial.



There are several paths
 for future research:
\begin{enumerate}

%Not true
%\item
%{ The Hurwitz integers have
% 24 units, so it may be possible to represent any Pythagorean 
%quintuple as a product of two quaternions and a unit. 
%}

%
%\item 
%{ Why can almost every 
%quintuple be generated
%in a multiple of 12 times? I suspect that this is an artifact
%of my program. 
%}


\item
{Conway and Smith have developed connections between the
 quaternions and 
orthogonal groups (rotations of space). 
Once I understand
 more group theory, their ideas may provide an important
 geometrical interpretation. 
}


\item
{ Perhaps the obvious geometrical analogue of Pythagorean quintuples
  --- the volumes of the facets of what we will call a
  tetrarectangular 4-dimensional simplex --- has a deeper meaning.  }


%\item
%{ Perhaps the obvious geometrical
%analogue of Pythagorean quintuples --- sections of a 4-dimensional
% rectangular prism, just as a right triangle is a section of
% a rectangle --- has deeper meaning.   
%}

\item

{In the 1930s, B.\ Berggren \cite{Swedish_Tree} discovered a set of
transformations that iteratively generate all primitive Pythagorean
triples.  F.\ J.\ M.\ Barning \cite{Dutch_Tree} independently found
this method and reformulated it into a set of matrices, and Conrad 
\cite{Conrad} provided a geometric interpretation.  Notably,
these transformations can be shown as the Barning-Hall tree, which
contains all primitive Pythagorean triples.  This approach is 
completely different from the methods in this paper, but
I plan to explore this path.}

%Berggren was not just republishing a result by H. Rath.  Rath appears
%to have compiled every known result on Pythagorean triples and not
%cited his sources.  https://hdl.handle.net/2027/mdp.39015036980533

%This paper seems like real math, but is not related enough to my
%project. It has a correct overview of the topic: A Discussion on
%Primitive Pythagorean Triples and Primitive Pythagorean Primes
%Duvvuri Surya Rahul and Snehanshu Saha


\end{enumerate}




\begin{appendices}
%%\appendix

\section{Detailed Analysis of a Pythagorean Quintuple That Does Not Arise from
Theorem~\ref{theorem_as_products}}
\label{appendix_A}



Consider the Pythagorean quintuple
$1^2 + 2^2 + 8^2 + 10^2 = 13^2$.
It
cannot be represented as in
Theorem~\ref{theorem_as_products}.
If it could, it would use one of the following generating lists.
The
coefficients are the only Hurwitz integers to have a modulus of 13:



\begin{enumerate}

\item $ \frac{7}{2},\frac{1}{2}, \frac{1}{2}, \frac{1}{2}  $\\
How could this list create the 1?
The terms that could add to 1 are
\begin{itemize}
\item $ \pm\frac{49}{4},\pm\frac{1}{4}, \pm\frac{1}{4}, \pm\frac{1}{4}  $  
\item or $ \pm\frac{7}{4},\pm\frac{7}{4}, \pm\frac{1}{4}, \pm\frac{1}{4}  $. 
\end{itemize}
Neither of these possibilities work.


\item $3,2,0,0$\\
How could this list create the 10?
The terms that could add to 10 are
\begin{itemize}
\item$ \pm9,\pm4, 0, 0  $,
\item$ \pm6,\pm6, 0, 0  $,
\item$ \pm6,0, 0, 0  $,
\item$ \pm4,0, 0, 0  $,
\item$ 0,0, 0, 0  $,
\item or $ \pm9,0, 0, 0  $.
\end{itemize}
None of these possibilities work.


\item $ \frac{5}{2},\frac{3}{2}, \frac{3}{2}, \frac{3}{2}  $\\
How could this list create the 1?
The terms that could add to 1 are
\begin{itemize}
\item$ \pm\frac{25}{4},\pm\frac{9}{4}, \pm\frac{9}{4}, \pm\frac{9}{4}  $  
\item or $ \pm\frac{15}{4},\pm\frac{15}{4}, \pm\frac{9}{4}, \pm\frac{9}{4}  $. 
\end{itemize}
Neither of these possibilities work.



\item $ \frac{5}{2},\frac{5}{2}, \frac{1}{2}, \frac{1}{2}  $\\
How could this list create the 1?
The terms that could add to 1 are
\begin{itemize}
\item$ \pm\frac{25}{4},\pm\frac{25}{4}, \pm\frac{1}{4}, \pm\frac{1}{4}  $,
\item$ \pm\frac{25}{4},\pm\frac{5}{4}, \pm\frac{5}{4}, \pm\frac{1}{4}  $,
\item or $ \pm\frac{5}{4},\pm\frac{5}{4}, \pm\frac{5}{4}, \pm\frac{5}{4}  $. 
\end{itemize}
None of these possibilities work.

v\item $2,2,2,1$\\
How could this list create the 10?
The terms that could add to 10 are
\begin{itemize}
\item$ \pm4,\pm4, \pm4, \pm1  $  
\item or $ \pm4,\pm4, \pm2, \pm2  $.
\end{itemize}
Neither of these possibilities work.


\end {enumerate}


\section{An Elementary Way to Generate Pythagorean Quintuples}
\label{appendix_B}

This method was inspired by the method used in \cite{New_Path}, which 
dealt with Pythagorean Quadruples. I strongly suspect that this concept
is a consequence of \cite{Formalized_New_Path}.


Choose 3 nonnegative integers $a$, $b$, and $c$, excluding the case 
where two of them are odd and one is even.





Choose integers $p$ and $q$ that obey the following 
conditions:


\begin{itemize}
\item[(i)]
$p|(a^2 + b^2 + c^2)$ 


\item[(ii)]
$pq = a^2 + b^2 + c^2$ 



\item[(iii)]
$p \equiv q \pmod 2$



\end{itemize}




Let $d$ = $\frac{p-q}{2}$ and
$e$ = $\frac{p+q}{2}$. 
Note that if $p \not\equiv q \pmod 2$, then $d$ and $e$
would not be integers.

Then $a^2 + b^2 + c^2 + d^2 = e^2$. All Pythagorean quintuples can 
be generated in this way.


\section{Testing the formulas from Theorem~\ref{theorem_as_products}}
\label{appendix_C}






The author wrote
a Python program
to test how many Pythagorean quintuples does Theorem~\ref{theorem_as_products}
represent.
This program found all primitive Pythagorean 
quintuples with all legs less than or equal to $20$.
The program then takes a form, applies it
to a set of four integers or half integers, and records the resulting 
Pythagorean quintuple.   


\begin{example}
Example of a generating list: 
$$ 4, 4, 0, 1
$$ 
Example of a form:
$$
     (x+yi+zj+wk)(y+zi+wj-xk)
$$
The program applies the form: 
$$
     (4+4i+0j+1k)(4+0i+1j-4k)=20+15i+20j-8k
$$
The result:
$$
       8, 15, 20, 20
$$
The result is always a Pythagorean quintuple: 
$$8^2+ 15^2+ 20^2+ 20^2= 33^2$$
\end{example}









The author then
analyzed the forms to
see if they
capture all of the test set of primitive Pythagorean quintuples.
To select the forms in Figure~\ref{figure_57}
the author began with the quaternion
$(x + yi + zj + wk)^2$ with 
 $x$, $y$, $z$ and $w$ all positive.
Theorem~\ref{theorem_as_products} required 
permutations and sign changes within the factors.
Without loss of generality, the first factor is
of the form $(x \pm yi \pm zj \pm wk)$,
while the second factor has all possible permutations
and sign changes of
 $x$, $y$, $z$ and $w$ to produce
$(x' + y'i + z'j + w'k)$.
This gives $2^7(4!)$ possible expressions.


The author let $x = \pi$, $y = e$,
$z = \sqrt{2}$ and $w = \sqrt{5}$ and evaluated
each of the $2^7(4!)$ expressions.
Of these, there were $57$ different resulting values.
The author chose one representative expression for each of the
$57$ values. Notice that there exists a great deal of 
redundancy among these $57$ expressions; any $2$ expressions
characterize roughly the same set of primitive
Pythagorean quintuples, at least for small examples. Intuitively,
this redundancy arises because each expression describes a way in which 
a quaternion can be factored. 
By Theorem~\ref{theorem_unique_factorization}, a quaternion that 
represents a primitive Pythagorean quintuple can be factored in many 
ways, so many of these $57$ expressions characterize it.


\vanish{
%%% Old figure of Heinz 57
\begin{figure}
\begin{tabbing}
spacespacespacespacespacespacespacespacespace\=              \kill
1. $(x+yi+zj+wk)(x+yi+zj+wk)$ \>283\\
2. $(x+yi+zj+wk)(x+yi+wj+zk)$ \>349\\
3. $(x+yi+zj+wk)(x+zi+yj+wk)$ \>349\\
4. $(x+yi+zj+wk)(x+zi+wj+yk)$ \>361\\
5. $(x+yi+zj+wk)(x+wi+yj+zk)$ \>352\\
6. $(x+yi+zj+wk)(x+wi+zj+yk)$ \>349\\
7. $(x+yi+zj+wk)(y+xi+zj+wk)$ \>379\\
8. $(x+yi+zj+wk)(y+xi+wj+zk)$ \>283\\
9. $(x+yi+zj+wk)(y+zi+xj+wk)$ \>352\\
10. $(x+yi+zj+wk)(y+zi+wj+xk)$ \>349\\
11. $(x+yi+zj+wk)(y+wi+xj+zk)$ \>379\\
12. $(x+yi+zj+wk)(y+wi+zj+xk)$ \>393\\
13. $(x+yi+zj+wk)(z+xi+yj+wk)$ \>393\\
14. $(x+yi+zj+wk)(z+xi+wj+yk)$ \>349\\
15. $(x+yi+zj+wk)(z+yi+xj+wk)$ \>379\\
16. $(x+yi+zj+wk)(z+yi+wj+xk)$ \>352\\
17. $(x+yi+zj+wk)(z+wi+xj+yk)$ \>283\\
18. $(x+yi+zj+wk)(w+xi+zj+yk)$ \>352\\
19. $(x+yi+zj+wk)(w+yi+xj+zk)$ \>393\\
20. $(x+yi+zj+wk)(w+zi+xj+yk)$ \>349\\
21. $(x+yi+zj+wk)(w+zi+yj+xk)$ \>283\\
22. $(x+yi+zj+wk)(x+yi+zj-wk)$ \>321\\
23. $(x+yi+zj+wk)(x+yi+wj-zk)$ \>405\\
24. $(x+yi+zj+wk)(x+zi+yj-wk)$ \>405\\
25. $(x+yi+zj+wk)(x+zi+wj-yk)$ \>353\\
26. $(x+yi+zj+wk)(x+wi+yj-zk)$ \>353\\
27. $(x+yi+zj+wk)(x+wi+zj-yk)$ \>291\\
28. $(x+yi+zj+wk)(y+xi+zj-wk)$ \>405\\
29. $(x+yi+zj+wk)(y+xi+wj-zk)$ \>321\\
30. $(x+yi+zj+wk)(y+zi+xj-wk)$ \>353\\
31. $(x+yi+zj+wk)(y+zi+wj-xk)$ \>405\\
32. $(x+yi+zj+wk)(y+wi+xj-zk)$ \>291\\
33. $(x+yi+zj+wk)(z+xi+yj-wk)$ \>353\\
34. $(x+yi+zj+wk)(z+xi+wj-yk)$ \>291\\
35. $(x+yi+zj+wk)(z+yi+xj-wk)$ \>291\\
36. $(x+yi+zj+wk)(z+wi+xj-yk)$ \>5\\
37. $(x+yi+zj+wk)(z+wi+yj-xk)$ \>291\\
38. $(x+yi+zj+wk)(w+xi+yj-zk)$ \>405\\
39. $(x+yi+zj+wk)(w+yi+zj-xk)$ \>405\\
40. $(x+yi+zj+wk)(w+zi+xj-yk)$ \>291\\
41. $(x+yi+zj+wk)(w+zi+yj-xk)$ \>321\\
42. $(x+yi+zj-wk)(x+yi+zj+wk)$ \>321\\
43. $(x+yi+zj-wk)(x+zi+wj+yk)$ \>353\\
44. $(x+yi+zj-wk)(x+wi+yj+zk)$ \>353\\
45. $(x+yi+zj-wk)(y+zi+xj+wk)$ \>353\\
46. $(x+yi+zj-wk)(z+xi+yj+wk)$ \>353\\
47. $(x+yi+zj-wk)(z+wi+xj+yk)$ \>321\\
48. $(x+yi+zj-wk)(w+zi+yj+xk)$ \>321\\
49. $(x+yi+zj-wk)(x+yi+wj-zk)$ \>349\\
50. $(x+yi+zj-wk)(x+zi+yj-wk)$ \>349\\
51. $(x+yi+zj-wk)(x+wi+zj-yk)$ \>349\\
52. $(x+yi+zj-wk)(y+xi+zj-wk)$ \>379\\
53. $(x+yi+zj-wk)(y+zi+wj-xk)$ \>379\\
54. $(x+yi+zj-wk)(y+wi+xj-zk)$ \>349\\
55. $(x+yi+zj-wk)(z+xi+wj-yk)$ \>379\\
56. $(x+yi+zj-wk)(z+wi+yj-xk)$ \>349\\
57. $(x+yi+zj-wk)(w+xi+yj-zk)$ \>349
\end{tabbing}
\caption{The $57$ possible forms of a Pythagorean quintuple in its primitive Hurwitz form as seen in Theorem~\ref{theorem_as_products}.} 
\label{figure_57}
\end{figure}
}



There are $337$ Pythagorean quintuples 
with all legs less than or equal to $20$.
See Figure~\ref{figure_quintuples}
for $50$ of those quintuples and
the number of ways to generate them:


\begin{figure}
$$
\begin{array}{l|c||l|c}
                  &\mbox{Times}    &   &\mbox{Times}\\ 
\mbox{Pythagorean Quintuple}     &\mbox{found}
&\mbox{Pythagorean Quintuple}     &\mbox{found}\\ \hline
&&&\\
0^2+ 0^2+ 0^2+ 1^2  = 1^2 & 255
& 2.5^2+ 3.5^2+ 6.5^2+ 19.5^2  = 21^2 &0\\
0^2+ 0^2+ 3^2+ 4^2  = 5^2 &180
&2.5^2+ 12.5^2+ 12.5^2+ 14.5^2  = 23^2 &36\\
%%
0^2+ 0^2+ 5^2+ 12^2  = 13^2 &180&
2.5^2+ 13.5^2+ 16.5^2+ 19.5^2  = 29^2 &0\\
%
0^2+ 0^2+ 8^2+ 15^2  = 17^2 &180&
3.5^2+ 5.5^2+ 8.5^2+ 10.5^2  = 15^2 &0\\
%
0^2+ 3^2+ 4^2+ 12^2  = 13^2 &252&
3.5^2+ 5.5^2+ 9.5^2+ 12.5^2  = 17^2 &0\\
%
1^2+ 2^2+ 8^2+ 10^2  = 13^2 &0&
3.5^2+ 5.5^2+ 15.5^2+ 18.5^2  = 25^2 &0\\
%
1^2+ 2^2+ 10^2+ 16^2  = 19^2 &0&
3.5^2+ 5.5^2+ 17.5^2+ 19.5^2  = 27^2 &0\\
%
1^2+ 4^2+ 4^2+ 4^2  = 7^2 &36&
3.5^2+ 6.5^2+ 6.5^2+ 8.5^2  = 13^2 &60\\
%
4^2+ 12^2+ 13^2+ 20^2  = 27^2 &0&
3.5^2+ 6.5^2+ 11.5^2+ 18.5^2  = 23^2 &0\\
%
4^2+ 13^2+ 16^2+ 20^2  = 29^2 &144&
3.5^2+ 7.5^2+ 10.5^2+ 10.5^2  = 17^2 &36\\
%
4^2+ 16^2+ 17^2+ 20^2  = 31^2 &0&
3.5^2+ 7.5^2+ 10.5^2+ 13.5^2  = 19^2 &0\\
%
8^2+ 13^2+ 14^2+ 14^2  = 25^2 &36&
3.5^2+ 7.5^2+ 11.5^2+ 15.5^2  = 21^2 &24\\
%
10^2+ 10^2+ 19^2+ 20^2  = 31^2 &36&
3.5^2+ 8.5^2+ 8.5^2+ 11.5^2  = 17^2 &60\\
%
12^2+ 16^2+ 17^2+ 20^2  = 33^2 &288&
4.5^2+ 4.5^2+ 7.5^2+ 8.5^2  = 13^2 &84\\
%
13^2+ 16^2+ 20^2+ 20^2  = 35^2 &36&
4.5^2+ 4.5^2+ 10.5^2+ 14.5^2  = 19^2 &84\\
%
13^2+ 20^2+ 20^2+ 20^2  = 37^2 &36&
4.5^2+ 4.5^2+ 13.5^2+ 17.5^2  = 23^2 &36\\
%
0.5^2+ 0.5^2+ 0.5^2+ 0.5^2  = 1^2 &30&
8.5^2+ 15.5^2+ 17.5^2+ 18.5^2  = 31^2 &0\\
%
0.5^2+ 0.5^2+ 1.5^2+ 2.5^2  = 3^2 &84&
9.5^2+ 10.5^2+ 19.5^2+ 19.5^2  = 31^2 &36\\
%
0.5^2+ 0.5^2+ 2.5^2+ 6.5^2  = 7^2 &84&
9.5^2+ 12.5^2+ 14.5^2+ 16.5^2  = 27^2 &48\\
%
0.5^2+ 0.5^2+ 3.5^2+ 3.5^2  = 5^2 &36&
10.5^2+ 10.5^2+ 11.5^2+ 16.5^2  = 25^2 &36\\
%
0.5^2+ 0.5^2+ 3.5^2+ 12.5^2  = 13^2 &84&
10.5^2+ 12.5^2+ 12.5^2+ 17.5^2  = 27^2 &36\\
%
0.5^2+ 0.5^2+ 5.5^2+ 9.5^2  = 11^2 &36&
10.5^2+ 16.5^2+ 16.5^2+ 17.5^2  = 31^2 &36\\
%
1.5^2+ 8.5^2+ 13.5^2+ 16.5^2  = 23^2 &0&
11.5^2+ 16.5^2+ 18.5^2+ 18.5^2  = 33^2 &36\\
%
1.5^2+ 10.5^2+ 10.5^2+ 17.5^2  = 23^2 &36&
12.5^2+ 12.5^2+ 17.5^2+ 18.5^2  = 31^2 &84\\
%
1.5^2+ 10.5^2+ 15.5^2+ 16.5^2  = 25^2 &0&
12.5^2+ 14.5^2+ 18.5^2+ 19.5^2  = 33^2 &48
\end{array}
$$
\caption{Generating program results for some small Pythagorean quintuples,}
\label{figure_quintuples}
\end{figure}


\vanish{
%\begin{figure}
\begin{tabbing}
spacespacespacespacespacespacespace\=    \kill
Pythagorean Quintuple     \>Number of ways to generate\\
$0^2+ 0^2+ 0^2+ 1^2  = 1^2$ \>$255$\\
$0^2+ 0^2+ 3^2+ 4^2  = 5^2$ \>$180$\\
$0^2+ 0^2+ 5^2+ 12^2  = 13^2$ \>$180$\\
$0^2+ 0^2+ 8^2+ 15^2  = 17^2$ \>$180$\\
$0^2+ 3^2+ 4^2+ 12^2  = 13^2$ \>$252$\\
$1^2+ 2^2+ 8^2+ 10^2  = 13^2$ \>$0$\\
$1^2+ 2^2+ 10^2+ 16^2  = 19^2$ \>$0$\\
$1^2+ 4^2+ 4^2+ 4^2  = 7^2$ \>$36$\\
$4^2+ 12^2+ 13^2+ 20^2  = 27^2$ \>$0$\\
$4^2+ 13^2+ 16^2+ 20^2  = 29^2$ \>$144$\\
$4^2+ 16^2+ 17^2+ 20^2  = 31^2$ \>$0$\\
$8^2+ 13^2+ 14^2+ 14^2  = 25^2$ \>$36$\\
$10^2+ 10^2+ 19^2+ 20^2  = 31^2$ \>$36$\\
$12^2+ 16^2+ 17^2+ 20^2  = 33^2$ \>$288$\\
$13^2+ 16^2+ 20^2+ 20^2  = 35^2$ \>$36$\\
$13^2+ 20^2+ 20^2+ 20^2  = 37^2$ \>$36$\\
$0.5^2+ 0.5^2+ 0.5^2+ 0.5^2  = 1^2$ \>$30$\\
$0.5^2+ 0.5^2+ 1.5^2+ 2.5^2  = 3^2$ \>$84$\\
$0.5^2+ 0.5^2+ 2.5^2+ 6.5^2  = 7^2$ \>$84$\\
$0.5^2+ 0.5^2+ 3.5^2+ 3.5^2  = 5^2$ \>$36$\\
$0.5^2+ 0.5^2+ 3.5^2+ 12.5^2  = 13^2$ \>$84$\\
$0.5^2+ 0.5^2+ 5.5^2+ 9.5^2  = 11^2$ \>$36$\\
$1.5^2+ 8.5^2+ 13.5^2+ 16.5^2  = 23^2$ \>$0$\\
$1.5^2+ 10.5^2+ 10.5^2+ 17.5^2  = 23^2$ \>$36$\\
$1.5^2+ 10.5^2+ 15.5^2+ 16.5^2  = 25^2$ \>$0$\\
$2.5^2+ 3.5^2+ 6.5^2+ 19.5^2  = 21^2$ \>$0$\\
$2.5^2+ 12.5^2+ 12.5^2+ 14.5^2  = 23^2$ \>$36$\\
$2.5^2+ 13.5^2+ 16.5^2+ 19.5^2  = 29^2$ \>$0$\\
$3.5^2+ 5.5^2+ 8.5^2+ 10.5^2  = 15^2$ \>$0$\\
$3.5^2+ 5.5^2+ 9.5^2+ 12.5^2  = 17^2$ \>$0$\\
$3.5^2+ 5.5^2+ 15.5^2+ 18.5^2  = 25^2$ \>$0$\\
$3.5^2+ 5.5^2+ 17.5^2+ 19.5^2  = 27^2$ \>$0$\\
$3.5^2+ 6.5^2+ 6.5^2+ 8.5^2  = 13^2$ \>$60$\\
$3.5^2+ 6.5^2+ 11.5^2+ 18.5^2  = 23^2$ \>$0$\\
$3.5^2+ 7.5^2+ 10.5^2+ 10.5^2  = 17^2$ \>$36$\\
$3.5^2+ 7.5^2+ 10.5^2+ 13.5^2  = 19^2$ \>$0$\\
$3.5^2+ 7.5^2+ 11.5^2+ 15.5^2  = 21^2$ \>$24$\\
$3.5^2+ 8.5^2+ 8.5^2+ 11.5^2  = 17^2$ \>$60$\\
$4.5^2+ 4.5^2+ 7.5^2+ 8.5^2  = 13^2$ \>$84$\\
$4.5^2+ 4.5^2+ 10.5^2+ 14.5^2  = 19^2$ \>$84$\\
$4.5^2+ 4.5^2+ 13.5^2+ 17.5^2  = 23^2$ \>$36$\\
$8.5^2+ 15.5^2+ 17.5^2+ 18.5^2  = 31^2$ \>$0$\\
$9.5^2+ 10.5^2+ 19.5^2+ 19.5^2  = 31^2$ \>$36$\\
$9.5^2+ 12.5^2+ 14.5^2+ 16.5^2  = 27^2$ \>$48$\\
$10.5^2+ 10.5^2+ 11.5^2+ 16.5^2  = 25^2$ \>$36$\\
$10.5^2+ 12.5^2+ 12.5^2+ 17.5^2  = 27^2$ \>$36$\\
$10.5^2+ 16.5^2+ 16.5^2+ 17.5^2  = 31^2$ \>$36$\\
$11.5^2+ 16.5^2+ 18.5^2+ 18.5^2  = 33^2$ \>$36$\\
$12.5^2+ 12.5^2+ 17.5^2+ 18.5^2  = 31^2$ \>$84$\\
$12.5^2+ 14.5^2+ 18.5^2+ 19.5^2  = 33^2$ \>$48$\\
\end{tabbing}
%\label{example_quintuples}
%\end{figure}
}


\end{appendices}



\newpage

\newcommand{\journal}[6]{{\sc #1,} #2, {\it #3} {\bf #4} (#5), #6.}
\newcommand{\preprint}[3]{{\sc #1,} #2, preprint #3.}
\newcommand{\book}[4]{{\sc #1,} #2, #3, #4.}
\newcommand{\collection}[6]{{\sc #1,}  #2, #3, in {\it #4}, #5, #6.}
\newcommand{\JCTA}{J.\ Combin.\ Theory Ser.\ A}
\newcommand{\arxiv}[3]{{\sc #1,} #2, {\tt #3}.}
\newcommand{\article}[3]{{\sc #1,} #2, {\tt #3}.}
\newcommand{\journalfive}[5]{{\sc #1,} #2, {\it #3}  (#4), #5.}





\begin{thebibliography}{1}

%\bibitem{Hardy_and_Wright}
%\book{G.\ H.\ Hardy and E.\ M.\ Wright}
%         {An Introduction to the Theory of Numbers, 6th Edition}
%         {Oxford University Press, Oxford} 
%         {2008}



\bibitem{Davidoff_Sarnak_Valette}
\book{G.\ Davidoff, P.\ Sarnak, and A.\ Valette}
         {Elementary Number Theory,
           Group Theory,
           and Ramanujan Graphs}
         {Cambridge University Press}
         {2003}

\bibitem{Dickson}
\book{L.\ E.\ Dickson}
         {History of the Theory of Numbers}
         {AMS Chelsea Publishing}
         {1999}


%\bibitem{Jacobi}
%\journal{C.\ G.\ J.\ Jacobi}
%        {De compositione numerorum e quatuor quadratis}
%        {Journal f\"ur die reine und angewandte Mathematik}
%        {12}{1834}{167--172}


\bibitem{Ferrari}
\journal{F.\ Ferrari}
        {Risoluzione Dell'Equazione}
        {Supplemento al Periodico di Matematica}
        {11}{1908}{129--131}



\bibitem{Mordell}
\book{L.\ J.\ Mordell}
         {Diophantine Equations}
         {Academic Press, London and New York} 
         {1969}


\end{thebibliography}






\end{document}




 

%\usepackage{abstract}
\usepackage{anyfontsize}
\usepackage{tikz}
\usepackage{amssymb, amsmath, fullpage, amsthm}
\usepackage{todonotes}

\parskip3mm

\newtheorem{theorem}{Theorem}
\newtheorem{property}[theorem]{Property}
\newtheorem{hypothesis}[theorem]{Hypothesis}
\newtheorem{lemma}[theorem]{Lemma}
\newtheorem{parametrization}[theorem]{Parametrization}
\newtheorem{proposition}[theorem]{Proposition}
\newtheorem{definition}[theorem]{Definition}
\newtheorem{corollary}[theorem]{Corollary}
\newtheorem{example}[theorem]{Example}
\newtheorem{remarks}[theorem]{Remarks}
\newtheorem{remark}[theorem]{Remark}

\numberwithin{equation}{section}
\usepackage[bindingoffset=0.2in,
            left=0.5in,right=0.5in,top=0.5in,bottom=0.5in,
            footskip=.25in]{geometry}%Sets margins
\pagenumbering{gobble}%No page numbers

\newcommand{\controlledbf}{}
\newcommand{\MySpacing}{25}
%\newcommand{\MySpacing}{10}
\newcommand{\MyFontSize}{20}
%\newcommand{\MyFontSize}{10}
\newcommand{\MySectionSpacing}{15}
%\newcommand{\MySectionSpacing}{10}
\newcommand{\MySectionFontSize}{40}
%\newcommand{\MySectionFontSize}{10}


%\renewcommand{\abstractnamefont}{\fontsize{\MyFontSize}{\MySpacing}}
%\renewcommand{\abstracttextfont}{\normalfont\small}


\DeclareMathOperator{\inv}{inv}
\DeclareMathOperator{\frst}{frst}
\DeclareMathOperator{\er}{er}
\DeclareMathOperator{\asc}{asc}
\DeclareMathOperator{\odd}{odd}

\newcommand{\ascodd}{\asc_{\odd}}

\newcommand{\doubleprime}{\prime\prime}


\font\Cp = msbm10

\newcommand{\Ccc}{\hbox{\Cp C}}
\newcommand{\Fff}{\hbox{\Cp F}}
\newcommand{\Hhh}{\hbox{\Cp H}}
\newcommand{\Nnn}{\hbox{\Cp N}}
\newcommand{\Rrr}{\hbox{\Cp R}}
\newcommand{\Sss}{\hbox{\Cp S}}
\newcommand{\Zzz}{\hbox{\Cp Z}}
%The above commands don't work with large font.

\newcommand{\SSSS}{\mathfrak{S}}

\newcommand{\Gaussian}[2]{\genfrac{[}{]}{0pt}{}{#1}{#2}_q}




\newcommand{\fix}[1]{\todo[inline]{#1}}

\begin{document}







\newpage










